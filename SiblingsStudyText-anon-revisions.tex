% Options for packages loaded elsewhere
\PassOptionsToPackage{unicode}{hyperref}
\PassOptionsToPackage{hyphens}{url}
%
\documentclass[
  man,mask,floatsintext]{apa6}
\usepackage{amsmath,amssymb}
\usepackage{iftex}
\ifPDFTeX
  \usepackage[T1]{fontenc}
  \usepackage[utf8]{inputenc}
  \usepackage{textcomp} % provide euro and other symbols
\else % if luatex or xetex
  \usepackage{unicode-math} % this also loads fontspec
  \defaultfontfeatures{Scale=MatchLowercase}
  \defaultfontfeatures[\rmfamily]{Ligatures=TeX,Scale=1}
\fi
\usepackage{lmodern}
\ifPDFTeX\else
  % xetex/luatex font selection
\fi
% Use upquote if available, for straight quotes in verbatim environments
\IfFileExists{upquote.sty}{\usepackage{upquote}}{}
\IfFileExists{microtype.sty}{% use microtype if available
  \usepackage[]{microtype}
  \UseMicrotypeSet[protrusion]{basicmath} % disable protrusion for tt fonts
}{}
\makeatletter
\@ifundefined{KOMAClassName}{% if non-KOMA class
  \IfFileExists{parskip.sty}{%
    \usepackage{parskip}
  }{% else
    \setlength{\parindent}{0pt}
    \setlength{\parskip}{6pt plus 2pt minus 1pt}}
}{% if KOMA class
  \KOMAoptions{parskip=half}}
\makeatother
\usepackage{xcolor}
\usepackage{graphicx}
\makeatletter
\def\maxwidth{\ifdim\Gin@nat@width>\linewidth\linewidth\else\Gin@nat@width\fi}
\def\maxheight{\ifdim\Gin@nat@height>\textheight\textheight\else\Gin@nat@height\fi}
\makeatother
% Scale images if necessary, so that they will not overflow the page
% margins by default, and it is still possible to overwrite the defaults
% using explicit options in \includegraphics[width, height, ...]{}
\setkeys{Gin}{width=\maxwidth,height=\maxheight,keepaspectratio}
% Set default figure placement to htbp
\makeatletter
\def\fps@figure{htbp}
\makeatother
\setlength{\emergencystretch}{3em} % prevent overfull lines
\providecommand{\tightlist}{%
  \setlength{\itemsep}{0pt}\setlength{\parskip}{0pt}}
\setcounter{secnumdepth}{-\maxdimen} % remove section numbering
% Make \paragraph and \subparagraph free-standing
\ifx\paragraph\undefined\else
  \let\oldparagraph\paragraph
  \renewcommand{\paragraph}[1]{\oldparagraph{#1}\mbox{}}
\fi
\ifx\subparagraph\undefined\else
  \let\oldsubparagraph\subparagraph
  \renewcommand{\subparagraph}[1]{\oldsubparagraph{#1}\mbox{}}
\fi
\newlength{\cslhangindent}
\setlength{\cslhangindent}{1.5em}
\newlength{\csllabelwidth}
\setlength{\csllabelwidth}{3em}
\newlength{\cslentryspacingunit} % times entry-spacing
\setlength{\cslentryspacingunit}{\parskip}
\newenvironment{CSLReferences}[2] % #1 hanging-ident, #2 entry spacing
 {% don't indent paragraphs
  \setlength{\parindent}{0pt}
  % turn on hanging indent if param 1 is 1
  \ifodd #1
  \let\oldpar\par
  \def\par{\hangindent=\cslhangindent\oldpar}
  \fi
  % set entry spacing
  \setlength{\parskip}{#2\cslentryspacingunit}
 }%
 {}
\usepackage{calc}
\newcommand{\CSLBlock}[1]{#1\hfill\break}
\newcommand{\CSLLeftMargin}[1]{\parbox[t]{\csllabelwidth}{#1}}
\newcommand{\CSLRightInline}[1]{\parbox[t]{\linewidth - \csllabelwidth}{#1}\break}
\newcommand{\CSLIndent}[1]{\hspace{\cslhangindent}#1}
\ifLuaTeX
\usepackage[bidi=basic]{babel}
\else
\usepackage[bidi=default]{babel}
\fi
\babelprovide[main,import]{english}
% get rid of language-specific shorthands (see #6817):
\let\LanguageShortHands\languageshorthands
\def\languageshorthands#1{}
% Manuscript styling
\usepackage{upgreek}
\captionsetup{font=singlespacing,justification=justified}

% Table formatting
\usepackage{longtable}
\usepackage{lscape}
% \usepackage[counterclockwise]{rotating}   % Landscape page setup for large tables
\usepackage{multirow}		% Table styling
\usepackage{tabularx}		% Control Column width
\usepackage[flushleft]{threeparttable}	% Allows for three part tables with a specified notes section
\usepackage{threeparttablex}            % Lets threeparttable work with longtable

% Create new environments so endfloat can handle them
% \newenvironment{ltable}
%   {\begin{landscape}\centering\begin{threeparttable}}
%   {\end{threeparttable}\end{landscape}}
\newenvironment{lltable}{\begin{landscape}\centering\begin{ThreePartTable}}{\end{ThreePartTable}\end{landscape}}

% Enables adjusting longtable caption width to table width
% Solution found at http://golatex.de/longtable-mit-caption-so-breit-wie-die-tabelle-t15767.html
\makeatletter
\newcommand\LastLTentrywidth{1em}
\newlength\longtablewidth
\setlength{\longtablewidth}{1in}
\newcommand{\getlongtablewidth}{\begingroup \ifcsname LT@\roman{LT@tables}\endcsname \global\longtablewidth=0pt \renewcommand{\LT@entry}[2]{\global\advance\longtablewidth by ##2\relax\gdef\LastLTentrywidth{##2}}\@nameuse{LT@\roman{LT@tables}} \fi \endgroup}

% \setlength{\parindent}{0.5in}
% \setlength{\parskip}{0pt plus 0pt minus 0pt}

% Overwrite redefinition of paragraph and subparagraph by the default LaTeX template
% See https://github.com/crsh/papaja/issues/292
\makeatletter
\renewcommand{\paragraph}{\@startsection{paragraph}{4}{\parindent}%
  {0\baselineskip \@plus 0.2ex \@minus 0.2ex}%
  {-1em}%
  {\normalfont\normalsize\bfseries\itshape\typesectitle}}

\renewcommand{\subparagraph}[1]{\@startsection{subparagraph}{5}{1em}%
  {0\baselineskip \@plus 0.2ex \@minus 0.2ex}%
  {-\z@\relax}%
  {\normalfont\normalsize\itshape\hspace{\parindent}{#1}\textit{\addperi}}{\relax}}
\makeatother

% \usepackage{etoolbox}
\makeatletter
\patchcmd{\HyOrg@maketitle}
  {\section{\normalfont\normalsize\abstractname}}
  {\section*{\normalfont\normalsize\abstractname}}
  {}{\typeout{Failed to patch abstract.}}
\patchcmd{\HyOrg@maketitle}
  {\section{\protect\normalfont{\@title}}}
  {\section*{\protect\normalfont{\@title}}}
  {}{\typeout{Failed to patch title.}}
\makeatother

\usepackage{xpatch}
\makeatletter
\xapptocmd\appendix
  {\xapptocmd\section
    {\addcontentsline{toc}{section}{\appendixname\ifoneappendix\else~\theappendix\fi\\: #1}}
    {}{\InnerPatchFailed}%
  }
{}{\PatchFailed}
\keywords{Siblings, Lexical Development, Input Effects, Language Acquisition}
\usepackage{lineno}

\linenumbers
\usepackage{csquotes}
\ifLuaTeX
  \usepackage{selnolig}  % disable illegal ligatures
\fi
\IfFileExists{bookmark.sty}{\usepackage{bookmark}}{\usepackage{hyperref}}
\IfFileExists{xurl.sty}{\usepackage{xurl}}{} % add URL line breaks if available
\urlstyle{same}
\hypersetup{
  pdftitle={Analysing the effect of sibling number on input and output in the first 18 months},
  pdflang={en-EN},
  pdfkeywords={Siblings, Lexical Development, Input Effects, Language Acquisition},
  hidelinks,
  pdfcreator={LaTeX via pandoc}}

\title{Analysing the effect of sibling number on input and output in the first 18 months}
\author{\textsuperscript{1} \& \textsuperscript{2}}
\date{}


\shorttitle{Effect of sibling number on language}

\authornote{

This work was funded by NIH Grant DP5-OD019812 to EB. Special thanks to Bergelson Lab trainees, staff, and RAs at Duke and Rochester (and especially Shannon Dailey, Andrei Amatuni and Hallie Garrison).
Supplementary materials, including code, for this manuscript can be found at \url{https://osf.io/rm3ec/?view_only=7cb606ed6805419b8b483e7c437e458f}.

Correspondence concerning this article should be addressed to , . E-mail:

}

\affiliation{\vspace{0.5cm}\textsuperscript{1} \\\textsuperscript{2} }

\abstract{%
Prior research suggests that across a wide range of cognitive, educational, and health-based measures, first-born children outperform their later-born peers. Expanding on this literature using naturalistic home-recorded data and parental vocabulary report, we find that early language outcomes vary by number of siblings in a sample of 43 English-learning U.S. children from mid-to-high socioeconomic status homes. More specifically, we find that children in our sample with two or more - but not one - older siblings had smaller productive vocabularies at 18 months, and heard less input from caregivers across several measures than their peers with less than two siblings. We discuss implications regarding what infants experience and learn across a range of family sizes in infancy.
}



\begin{document}
\maketitle

A common simplifying assumption in research on language development is that there is a theoretical ``optimum'' environment for early language, whereby the input is tailored to a single infant's needs, changing over time as language capacity grows (e.g. Soderstrom, 2007; Stern, Spieker, Barnett, \& MacKain, 1983). However, for many infants and for many reasons, language acquisition occurs across diverse social contexts that can influence the learning environment, including the presence of older siblings in the home (Fenson et al., 1994). According to the United States Census Bureau (2010), around one third of children are born into households with at least one other infant present, and one in every five infants is acquiring language in a household shared with two or more other children. Similar statistics are reported for British infants (Office for National Statistics, 2018), where the average household has 1.75 children, and 15\% of households have three children or more. In this paper, we consider the role of siblings in the early language environment of English-learning infants. We use naturalistic home-recorded data to measure input in earlier- and later-born infants, alongside their productive vocabulary over the first 18 months of life.

Prior research suggests that infants born to households with older children may be slower to learn language. Fenson and colleagues (1994) found that by 30 months of age, children with older siblings performed worse than those with no siblings across parent-reported measures of productive vocabulary, use of word combinations, and mean length of utterance. This `sibling effect' may be the result of differences in input between first- and later-born children: some research finds that infants with older siblings hear less speech aimed specifically at them, and what they do hear is understood to be linguistically less supportive of early language development (Hoff-Ginsberg, 1998; Oshima-Takane \& Robbins, 2003). In contrast, some studies have noted linguistic \emph{advantages} for later-borns, who may have stronger social-communicative skills (Hoff, 2006), better understanding of pronouns (Oshima-Takane, Goodz, \& Derevensky, 1996), and better conversational abilities (Dunn \& Shatz, 1989). Overall, while the particulars differ across studies, prior work suggests that the presence of siblings in the home leads to differences in infants' early linguistic experiences and skills, though the direction of these differences varies depending on what aspects of language are being measured.

Numerous studies have attempted to better understand how siblings affect the language development trajectory, with comparisons of language acquisition across first- and later-borns. Here again, findings are mixed, but overall two general conclusions can be drawn. First, analyses consistently show that infants with older siblings generally have slower \emph{vocabulary development} (Berglund, Eriksson, \& Westerlund, 2005; Fenson et al., 1994; Pine, 1995; Zambrana, Ystrom, \& Pons, 2012), and this effect increases with number of older siblings (Gurgand et al., 2022; Karwath, Relikowski, \& Schmitt, 2014; Peyre et al., 2016). Furthermore, this finding is consistent across cultures (e.g.~European French (Gurgand et al., 2022; Havron et al., 2019); Singaporean (Havron et al., 2022); Kenyan (Jakiela, Ozier, Fernald, \& Knauer, 2020); and German (Karwath et al., 2014)). However, this finding is not as clear-cut as has been previously assumed: Hoff-Ginsberg (1998) shows first-borns to have better lexical and syntactic skills up until 2;5, but later-born infants had better conversational abilities during the same time-period. Recent studies have also identified effects for age gap between the target child and their siblings (whereby larger age gaps correlate with lower vocabulary scores (Gurgand et al., 2022; Havron et al., 2022)) and sibling sex (whereby older brothers have a negative effect on vocabulary outcomes, but not older sisters (Havron et al., 2019; Jakiela et al., 2020)) though neither of these effects are found consistently across datasets; Havron et al. (2022) and Gurgand et al. (2022) find no effect for sibling sex, whereas Havron et al. (2019) find no effect for age gap. Some of these differences across studies may relate to insufficient power to detect relatively small effects, perhaps leading to under- or over-estimation of effect sizes, or simultaneous contributing factors that are difficult to disentangle.

The second general finding pertains to sibling-related differences in the early linguistic \emph{environment}: infants with no siblings receive more input overall, and this more closely reflects what is typically considered to be `high quality' input in the extant literature (i.e.~more input in an infant-directed speech style (Ramírez-Esparza, García-Sierra, \& Kuhl, 2014); longer utterance length (Barnes, Gutfreund, Satterly, \& Wells, 1983); higher lexical diversity (Rowe \& Snow, 2020)). Indeed, the very presence of a sibling in the linguistic environment changes the way language is used. When siblings are present (i.e.~triadic interactions), mothers' input has been found to be more focused on regulating behaviour, as opposed to the language-focused speech that is common in dyadic contexts (Oshima-Takane \& Robbins, 2003). Reports show that the mean length of utterance is longer in the input of first-born infants (Hoff-Ginsberg, 1998; but see also Oshima-Takane \& Robbins, 2003 for a comparison of dyadic and triadic contexts), who also hear more questions directed at them than later-borns. Both Jones and Adamson (1987) and Oshima-Takane and Robbins (2003) report no difference between the overall number of word types produced by mothers in dyadic and triadic settings, but the proportion of speech directed at the target infant is drastically reduced when input is shared with siblings.

As Hoff (2006) explains, infants with siblings have less experience of speech directed at them, but they do have an advantage over their first-born peers in that they are subject to more overheard speech. This may be an important source of input for infants with one or more older siblings. Akhtar, Jipson and Callanan (2001) show that, by age 2;6, infants can learn both novel object labels and novel verbs through overhearing. Slightly younger infants (aged 1;11-2;2) were also able to learn the novel object labels, but not verb labels. Two-year-old infants can even learn novel object labels while doing activities that distract them from the language input, and when the novel words are produced non-saliently (Akhtar, 2005). This suggests that, while the learning environment for later-borns might differ from that of first-born infants, there may be ample opportunity for them to learn from the speech that surrounds them; namely overheard speech directed at their older sibling(s). Evidence is mainly drawn from work testing infants aged 2 and above (e.g. Akhtar, 2005; Fitch, Liberman, Luyster, \& Arunachalam, 2020; Foushee, Srinivasan, \& Xu, 2021), and generally relies on experimental work rather than observations of the home environment. However, Floor and Akhtar (2006) tested younger infants to find that the capacity to learn from overheard speech is available from as early as 16 months, at least in an experimental setting.

There thus may be a trade-off, even in early development, between highly supportive one-to-one input from a caregiver (cf. Ramírez-Esparza et al., 2014) and the potential benefits drawn from communicating with (or overhearing communication with) a sibling. In the present study, we test the extent to which having more versus fewer siblings in the home environment may affect the linguistic environment in ways that could lead to differences in vocabulary development over the course of the first 18 months of life. In analyzing infants' growing productive vocabulary and linguistic environment in relation to the presence of older siblings in their household, the present work expands on the extant literature in two key ways. First, much of the existing literature identifying links between sibling number and vocabulary outcomes draws on large-scale questionnaire data, rather than naturalistic day-to-day interactions in the home. In contrast, we analyze an existing corpus of home recordings in concert with vocabulary checklists, in order to capture the reality of the early linguistic environment and how this is affected by sibling number. Second, we consider the opportunities that overheard speech might present in the infant's linguistic environment. We examine the effect of sibling number on overall amount of input produced in our naturalistic recordings, as well as, crucially, the extent to which parents label objects being attended to by the infant (\emph{object presence}\footnote{We’ve retained the term object presence for continuity with prior work using this variable but note that what this variable captures isn’t merely whether the object was present but rather whether it was present when the word for it was said aloud}). The analysis of object presence will allow us to gain insight into the kinds of learning opportunities being presented to infants in the early input, based on the previous research showing that object labeling - even when not directed specifically at the target child - can be a valuable source for acquiring linguistic knowledge. Based on work summarized above, we expect that both the language environment and infants' early productive vocabulary will vary as a function of how many older siblings they have.

\hypertarget{hypotheses}{%
\subsection{Hypotheses}\label{hypotheses}}

Synthesizing the work above in broad strokes, given prior research showing that early lexical development is more advanced among first-born infants (e.g. Hoff-Ginsberg, 1998), we predict that children with more siblings will have lower productive vocabularies than their peers with fewer siblings. However, we have no \emph{a priori} predictions about how these differences will manifest gradiently (e.g.~linear decrease for each additional sibilng, a threshold effect where we see a drop after a certain sibship size, etc.). With regard to the infants' linguistic environment, we hypothesize that infants with more siblings will experience lower prevalence of two aspects of the language input previously shown to support language development: \textbf{amount of input} and \textbf{amount of object presence.} Just as for productive vocabulary size, we do not make \emph{a priori} predictions regarding the shape of these effects, beyond predicting a decrease with sibling number. More specifically regarding input, following previous studies that show infants with siblings to receive less speech directed at them (Jones \& Adamson, 1987; Oshima-Takane \& Robbins, 2003), we expect to see the same effect in our sample. As noted above, by object presence we mean word and object co-occurrence, e.g.~mother saying ``cat'' when the child is looking at a cat. We predict object presence will decrease as sibling number increases, because as caregivers' attention is drawn away from one-to-one interactions with the infant, there is likely less opportunity for contingent talk and joint attention. Prior research suggests links between object presence and early word learning (Bergelson \& Aslin, 2017; Cartmill et al., 2013), though to our knowledge this has not been examined in relation to sibling number.

\hypertarget{methods}{%
\section{Methods}\label{methods}}

We analyze data from the SEEDLingS corpus, a longitudinal set of data incorporating home recordings, parental reports and experimental studies from the ages of 0;6 to 1;6. See Bergelson, Amatuni, Dailey, Koorathota, \& Tor (2019) for further details on the full set of home-recorded data and its annotations. The present study draws on the parental report data to index child vocabulary size, and annotations of hour-long home video recordings, taken on a monthly basis during data collection, to index input.\footnote{We also ran our input analysis using data sub-sampled from day-long audio recordings taken on a different day from the video data reported below; results were consistent with those outlined below for most analyses (see Supplementary Materials, S1).} We note at the outset that with such a multidimensional dataset there are always alternative ways of conducting analyses of input and output; due to limited power in our sample, we are unable to consider all potential variables (note that the same data was analysed in a previous study to find that mothers' work schedules could be related to vocabulary knowledge at 17 months (Laing \& Bergelson, 2019); we do not take this finding into account here). Our goal here is to make motivated decisions that we clearly describe, provide some alternative analytic choices in the supplementals, and to share the data with readers such that they are free to evaluate alternative approaches.

\hypertarget{participants}{%
\subsection{Participants}\label{participants}}

Forty-four families in New York State completed the year-long study. Infants (21 females) were from largely middle-class households; 33 mothers had attained a B.A. degree or higher. Based on parental report, no infants had speech- or hearing-relevant diagnoses; none were low birth weight (all \textgreater2,500g); 42 were white, two were from multi-racial backgrounds. All infants heard \textgreater75\% English on a regular basis and lived in two parent homes. Two participants were dizygotic twins; we retain one twin in the current sample, considering the other only as a sibling\footnote{Results were consistent when both twins were removed from the dataset, see S2.}. Thus our final sample size was 43 infants.

\hypertarget{sibling-details}{%
\subsubsection{Sibling Details}\label{sibling-details}}

Sibling number was computed based on parental report in the demographics questionnaires completed at 0;6 (Sibling number range: 0-4). Siblings were on average 4.11 years older than the infants in this study (SD: 4.01 years, R: 0-17 years).\footnote{For six infants, siblings' exact birthdates were not provided, and so age difference was estimated by subtracting the infant's age (6 months) from the sibling's age in years, as listed on the questionnaire (e.g.~if a sibling was 5 years old, they were classed as being 4.5 years older than the infant).} All siblings lived in the household with the infant full time, apart from one infant who had two older half siblings (and no other full siblings) who lived with their other parent part of the time. Both older siblings were present for at least some of the monthly recordings. One family had a foster child live in the home for 2 months of data collection, who is not accounted for in our data; the target infant had one sibling. All siblings were older than or of the same age as the infant in question.

\hypertarget{materials}{%
\subsection{Materials}\label{materials}}

\hypertarget{parental-report-data}{%
\subsubsection{Parental report data}\label{parental-report-data}}

To index each child's language abilities, we draw on data from vocabulary checklists (MacArthur-Bates Communicative Development Inventory, hereafter CDI, Fenson et al. 1994), administered monthly from 0;6 to 1;6, along with a demographics questionnaire; each month's CDI survey came pre-populated with the previous month's answers to save on reduplicated effort. Because the majority of infants did not produce their first word until around 0;11 according to CDI reports (M=10.67, SD=2.23), we use CDI data from 0;10 onwards in our analysis. CDI production data for each month is taken as a measure of the infants' lexical development. CDI data for production has been well validated by prior work, including work in this sample (Frank, Braginsky, Yurovsky, \& Marchman, 2021; Moore, Dailey, Garrison, Amatuni, \& Bergelson, 2019). Of the intended 13 CDIs collected for each of the 43 infants, 26 were missing across 11 infants (leaving 559 CDIs in total). 4 infants had 4 CDI data-points missing, while the majority (n = 5) had only one missing data-point.

\hypertarget{home-recorded-video-data}{%
\subsubsection{Home-recorded video data}\label{home-recorded-video-data}}

Every month between 0;6 and 1;5, infants were video-recorded for one hour in their home, capturing a naturalistic representation of each infant's day-to-day input. We did not ask families to ensure certain family members were or were not present; our video recordings capture whoever was home at the time families opted to schedule. Here we draw on data from the two caregivers who produced the most words in each recording; in 87\% of cases this was the mother, and 8\% of cases the father. Fathers produced the second highest number of words in 50\% of cases (see S3 for a full breakdown of speakers classed as caregivers in the dataset). At the child level, the modal caretaker across the 12 videos was the mother for 37 infants, father for 4 infants and grandmother for the remaining 2 infants. Infants wore a hat with two small Looxcie video cameras attached, one pointed slightly up, and one pointed slightly down; this captured the scene from the infants' perspective. In the event that infants refused to wear the hats, caregivers wore the same kind of camera on a headband. Additionally, a camcorder on a tripod was set up in the room where infants and caretakers were interacting to capture a broader view; families were asked to move this camcorder if they changed rooms. The dataset includes 12 videos for each child, one for each month that we analyzed.

Object words (i.e.~concrete nouns) deemed to be said to, by, or loudly and clearly near the target child were annotated by trained coders for several properties of interest to the broader project on noun learning. Here we examine annotations for speaker, i.e.~who produced each noun, and object presence, i.e.~whether the noun's referent was present and attended to by the infant (see ``Derived Input Measures'' below).

\hypertarget{derived-input-measures}{%
\paragraph{Derived Input measures}\label{derived-input-measures}}

Two input measures were derived based on the individual word level annotations of concrete nouns directed to or near the target child in this corpus, each pertaining to an aspect of the input that is established as important in early language learning: \textbf{overall household input} (how many concrete nouns does each infant hear? Note that this measure only includes speech produced directly to or close by the target child; see example below) and \textbf{object presence} (what proportion of this input is referentially transparent?), detailed below. Neither of these measures are, in our view, interpretable as ``pure'' quality or quantity input measures; we hold that quality and quantity are inextricably linked in general, and specifically we include (by design) only object words that the recordings suggest were possible learning instances for the infants who heard them, wherein quantity and quality are conflated. This included only concrete, imageable words that were addressed directly to the child (e.g.~``Have you got your toy \emph{bear}?''), or sufficiently loud and proximal that they were clearly audible to the child (e.g.~``Can you pass me the toy \emph{bear}?'', directed at the sibling while the infant sits nearby). As mentioned above, only speech produced in the infant's immediate surroundings (i.e.~speech that would have been clearly heard by the target infant) was coded.

\emph{Household Input} reflects how many nouns infants heard in the recordings from their two main caregivers (operationalized as the two adults who produced the most nouns in each recording; see above) and (where relevant) siblings. Input from speakers other than these two caregivers was relatively rare during video recordings, accounting for 0.39\% of input overall (SD=2.37\%), and is excluded from our analysis. This measure of the early language environment is based on evidence showing strong links between the amount of speech heard in the early input and later vocabulary size (Anderson, Graham, Prime, Jenkins, \& Madigan, 2021). This analysis considers only nouns produced by speakers in the child's environment, directed to or produced clearly near the child (which is what was annotated in the broader SEEDLingS project); concrete nouns are acquired earlier in development in English and cross-linguistically (Braginsky, Yurovsky, Marchman, \& Frank, 2019). As in any sample of naturalistic interaction, the number of nouns correlates highly with the number of words overall (e.g.~based on automated analyses of adult word counts vs.~manual noun-only annotations, Bulgarelli \& Bergelson, 2020). Thus, noun count in the monthly hour of video data serves as our household input proxy.

\emph{Object Presence} was coded as ``yes'', ``no'', or ``unsure'' for each object word annotated in the home recordings, as produced by the two main caregivers detailed above, based on trained annotators' assessment of whether the referent of the word (i.e.~the object) was present and attended to or touched by the child or the caregiver. For example, if the caregiver was pointing at a ball while the saying the word \emph{ball}, this was coded as ``yes''. If the infant was holding (but not looking at) a bottle while the caregiver said \emph{bottle}, this would also be coded as ``yes''. On the other hand, if the caregiver refers to a dog that is barking in the other room, that would be coded as ``no'', as it was not present during object labeling. In the video data, 182 instances (0.24\% on average per infant) of object presence were marked as unsure; these instances were not included in this analysis.

\hypertarget{data-analysis}{%
\subsection{Data analysis}\label{data-analysis}}

While we set out to test the hypotheses outlined above, aspects of our analysis were exploratory in nature. In respect of this, and on the advice of a helpful anonymous reviewer, we focus on descriptive and confirmatory measures of analysis through data visualization and effect size reporting alongside significance testing. For each of key variable tested, we present these three avenues for understanding the data, alongside any further follow-up exploratory analyses, where appropriate.

All reported models were generated in R (R Core Team, 2019) using the \emph{lmerTest} package to run linear mixed-effects regression models when needed(Kuznetsova, Brockhoff, \& Christensen, 2017). \emph{P}-values were generated by likelihood ratio tests resulting from nested model comparison. All models include infant as a random effect. All post-hoc tests are two-sample, two-tailed Wilcoxon Tests; given that all of our variables of interest (CDI score, household input and object presence) differed significantly from normal by Shapiro tests, we opted to run non-parametric tests for all post-hoc comparisons. Where multiple post-hoc comparisons are run on the same dataset, Bonferroni corrections are applied (e.g.~with an adjusted \emph{p}-value threshold of .025 for 2 between-group comparisons). While we have a substantial amount of data for each participant, our limited \emph{n} means we are under-powered to consider multiple demographic variables simultaneously given the data distribution (e.g.~sibling number and sex, see Table \ref{tab:table-sibling-number}; as luck would have it both infants with 3 siblings were girls and both with 4 were boys). There were no correlations between sibling number or child word production and maternal age/education. See S4 for further details.

\hypertarget{results}{%
\section{Results}\label{results}}

Our analyses consider infants' total productive vocabulary\footnote{While in principle we could have just used noun productive vocabulary, in practice noun and total vocabulary is correlated \textgreater.95 in this age range; we opted to retain the overall total vocabulary, as lexical class is not a straightforward notion in the early lexicon.} alongside our two input measures -- nouns in household input and extent of object presence in the input -- as a function of sibling number. Since the raw data are highly skewed, log-transformed data and/or proportions are used for statistical analysis (1 was added to the raw infant production data of all infants before log-transformation to retain infants with vocabularies of 0.) Unless otherwise specified, all figures display non-transformed data for interpretive ease.

Vocabulary development was highly variable across the 43 infants, according to the CDI data we had available. By 18 months, 2 infants produced no words (taken from 36 available CDIs at this time-point), while mean productive vocabulary size was 60.28 words (SD=78.31, Mdn=30.50). Three infants had substantially larger-than-average (3SDs above the monthly mean) vocabularies at certain time-points in the data; we counted one of these infants as an outlier and remove this child's data from the CDI analysis given that their vocabulary was higher for multiple consecutive months (1;1-1;6). The other two infants had higher vocabularies at 10-11 months only (when variance was quite limited, see Figure 1), and were retained to maximize data inclusion. This left 42 infants (19 females) in the analysis of vocabulary size. Infants had one sibling on average (M=0.86, Mdn=1, SD=1.10). See Table \ref{tab:table-sibling-number}.

\begin{table}[tbp]

\begin{center}
\begin{threeparttable}

\caption{\label{tab:table-sibling-number}Sibling number by female and male infants (n=42). One child was an outlier, and was removed from the CDI analysis and this table; see text for details.}

\small{

\begin{tabular}{llll}
\toprule
n Siblings & \multicolumn{1}{c}{Female} & \multicolumn{1}{c}{Male} & \multicolumn{1}{c}{Total}\\
\midrule
0 & 9 & 12 & 21\\
1 & 6 & 6 & 12\\
2 & 2 & 3 & 5\\
3 & 2 & 0 & 2\\
4 & 0 & 2 & 2\\
Total & 19 & 23 & 42\\
\bottomrule
\end{tabular}

}

\end{threeparttable}
\end{center}

\end{table}

\hypertarget{effect-of-siblings-on-infants-productive-vocabulary}{%
\subsection{Effect of siblings on infants' productive vocabulary}\label{effect-of-siblings-on-infants-productive-vocabulary}}

We first modeled the effect of siblings on reported productive vocabulary. We explored three possible variations on how to represent the sibling effect: a binary variable (0 vs.~\textgreater0 siblings), aggregated groups (None vs.~One vs.~2+ siblings), and discrete sibling number (0 to 4 siblings), comparing the following nested model structures, where (1) is the baseline model and (2) includes siblings as the variable of interest.

\begin{enumerate}
\def\labelenumi{\arabic{enumi}.}
\tightlist
\item
  vocabulary size (log-transformed) \textasciitilde{} age (months) + (1\textbar subject)
\item
  vocabulary size (log-transformed) \textasciitilde{} siblings {[}binary, group or discrete{]} + age (months) + (1\textbar subject)
\end{enumerate}

In our sample, simply having siblings (i.e.~as a binary variable) did not predict CDI productive vocabulary size, while both discrete sibling number and sibling group did. See Table \ref{tab:table-sibling-model-output}.\footnote{While our sample size and distribution leaves it statistically questionable to consider both sex and sibling number, for completeness we did also run a model that included sex in addition to age and sibling number (our primary variable of interest). Sex did not improve model fit over and above the effect of siblings in any of the three comparisons (\emph{p}s all \textgreater0.54).}

\begin{table}[H]

\begin{center}
\begin{threeparttable}

\caption{\label{tab:table-sibling-model-output}Output from likelihood ratio tests comparing regression models that predict the effects of sibling number (binary, grouped and discrete variables) on vocabulary size. Month was included in each model as a fixed effect; subject was included as a random effect.}

\small{

\begin{tabular}{lllll}
\toprule
Model & \multicolumn{1}{c}{Df} & \multicolumn{1}{c}{Chisq} & \multicolumn{1}{c}{p value} & \multicolumn{1}{c}{R2}\\
\midrule
0 vs. >0 siblings & 1.00 & 2.13 & 0.14 & 0.81\\
Sibling group & 2.00 & 8.00 & 0.02 & 0.81\\
Sibling number & 1.00 & 6.08 & 0.01 & 0.81\\
\bottomrule
\end{tabular}

}

\end{threeparttable}
\end{center}

\end{table}

\newpage

\begin{longtable}[t]{cccccc}
\caption{\label{tab:table-sibgroup-model-summary}Full model output from linear mixed effects regression models comparing language development over time in relation to sibling group. Age in months was included as a fixed effect; subject was included as a random effect.}\\
\toprule
Effect & Estimate & Std. Error & df & t value & p\\
\midrule
Intercept & -2.69 & 0.26 & 156.59 & -10.27 & <0.001\\
SibGroupOne & -0.01 & 0.30 & 42.08 & -0.05 & 0.963\\
SibGroup2+ & -0.94 & 0.33 & 42.84 & -2.81 & 0.007\\
month & 0.34 & 0.01 & 315.13 & 25.19 & <0.001\\
\bottomrule
\end{longtable}

Having more siblings was associated with a smaller vocabulary size over the course of early development. This is consistent with previous findings (Hoff-Ginsberg, 1998; Pine, 1995). We find that for each additional sibling, infants were reported to have produced 30.52\% fewer words. The `sibling effect' is thus present in our data.

In terms of our grouped sibling variable (i.e.~0 vs.~1 vs.~2+ siblings), infants with one sibling acquire only 1\% fewer words than firstborns over the course of our analysis, while infants with two or more siblings produce 94\% fewer words. See Table \ref{tab:table-sibgroup-model-summary} and and Figure \ref{fig:Figure-SibGroup}. Post-hoc Wilcoxon Rank Sum tests comparing reported productive vocabulary at 18 months (where there's the widest vocabulary range) revealed significantly larger vocabularies for infants with one sibling compared to those with two or more siblings (\emph{W}=5, \emph{p} = .004, CI={[}-72.00,-12.00{]}), but no difference between infants with one sibling and those with no siblings (\emph{W}=79.50, \emph{p} = .631, CI={[}-34.00,34.00{]}). See Table \ref{tab:table-data-summary}.

\begin{figure}
\centering
\includegraphics{SiblingsStudyText-anon-revisions_files/figure-latex/Figure-SibGroup-1.pdf}
\caption{\label{fig:Figure-SibGroup}Reported productive vocabulary acquisition (CDI) over time (n=42; one child was an outlier, and was removed from the CDI analysis and this figure; see text for details). Colors denote sibling group; line with colored confidence band reflects local estimator (loess) fit over individual infants' vocabulary at each month. Triangles indicate mean with bootstrapped CIs computed over each month's data. Points (jittered horizontally) show individual infants' vocabulary size at each month. Y-axis utilizes log-transformed vertical spacing for visual clarity.}
\end{figure}

\hypertarget{effect-of-siblings-on-infants-input}{%
\subsection{Effect of siblings on infants' input}\label{effect-of-siblings-on-infants-input}}

Having established that infants' productive vocabulary varied as a function of sibling number in all but the binary version of the measure (0 vs.~\textgreater0 siblings), we turn to our input measures to test whether \emph{input} varied by a child's sibling status. For these analyses we report here the group sibling division (0 vs.~1 vs.~2+) as this lets us keep relatively similar Ns across groups, thus making variance more comparable for post-hoc comparisons (the discrete sibling number (0-4) version is reported for completeness in S5; results hold for both input variables). We also now include the child who was a multi-month vocabulary outlier above, given that input and vocabulary are not tested in the same model. One infant of the full sample of 43 infants was an outlier in that they heard substantially more input words and words with object presence than all the other infants in the sample in four of their recording sessions. Given that these sessions were not contiguous, we opted to keep this infant in the analyses reported below, though all results hold when they are removed from our sample (see S6).

While we didn't have strong a priori expectations about how overall input or object presence would vary by age or sex, these were included in initial model comparisons to see if they improved fit alongside a random effect of infant. Both variables improved fit for the input model, and only age did for the object presence model. Thus our baseline models include these sets of control variables, respectively. See Table \ref{tab:table-input-model-summary} for final model estimates.

\hypertarget{caregiver-input}{%
\subsubsection{Caregiver Input}\label{caregiver-input}}

We tested overall quantity of input (aggregated across the two main caregivers, as outlined above, and siblings) in our model alongside age, sex and subject, as noted above, and a significant effect was found for the effect of sibling group (\(\chi^2 (2)\) = 8.88, \emph{p}=.012, \(R^2\)=0.59). Infants with one sibling heard on average 1\% more words than those with no siblings in any given hour-long recording, while infants with two or more siblings heard 49\% fewer words.

We then ran post-hoc tests to compare mean amount of input across sibling groups; these showed a significant difference in average input received between infants with one sibling versus those with two or more siblings (\emph{W}=11, \emph{p}=.002, CI={[}-120.87,-38.75{]}; Bonferroni-corrected \emph{p}-threshold = .025 for all reported Wilcoxon tests) while amount of input did not differ between infants with no siblings and those with one sibling (\emph{W}=146, \emph{p} = .736, CI={[}-39.50,57.50{]}). See Table \ref{tab:table-data-summary} for overall group differences (M and SD) in amount of input.

While we operationalized caregiver input in our models as input speech from the two adults who produced the most words in any given session, in 86.73\% of cases this was the mother or father. Considering mothers and father specifically, maternal input accounted for 75\% of object words in the data overall (M=195.49 words, Mdn=162.31, SD=108.93)\footnote{One family in our sample had two mothers; rather than artificially assigning one parent to another category, we averaged both mothers' input for this child; we acknowledge that this is an imperfect solution but found it better than the alternatives.}. Fathers accounted for an average of 18\% (M=58.72, Mdn=33, SD=65.05), while infants with siblings received around 12\% of their input from their brothers and sisters (M=22.97, Mdn=18, SD=18.49). See Table \ref{tab:table-data-summary} and Figure \ref{fig:Figure-Speaker-count}. As well as the overall input being greater for firstborns, compared with infants with one or 2+ siblings, note also that the variance is greater for this group, and decreases as sibling number increases. This is shown in the SDs reported in Table \ref{tab:table-data-summary}, and in the data points visualized in Figure \ref{fig:Figure-Speaker-count}.

\begin{table}[!h]

\caption{\label{tab:table-data-summary}Data summary of our two input measures and reported vocabulary size at 18 months.}
\centering
\begin{tabular}[t]{ccccccc}
\toprule
\multicolumn{1}{c}{ } & \multicolumn{2}{c}{No siblings} & \multicolumn{2}{c}{1 sibling} & \multicolumn{2}{c}{2+ siblings} \\
\cmidrule(l{3pt}r{3pt}){2-3} \cmidrule(l{3pt}r{3pt}){4-5} \cmidrule(l{3pt}r{3pt}){6-7}
\multicolumn{1}{c}{Variable} & \multicolumn{1}{c}{Mean} & \multicolumn{1}{c}{SD} & \multicolumn{1}{c}{Mean} & \multicolumn{1}{c}{SD} & \multicolumn{1}{c}{Mean} & \multicolumn{1}{c}{SD} \\
\cmidrule(l{3pt}r{3pt}){1-1} \cmidrule(l{3pt}r{3pt}){2-2} \cmidrule(l{3pt}r{3pt}){3-3} \cmidrule(l{3pt}r{3pt}){4-4} \cmidrule(l{3pt}r{3pt}){5-5} \cmidrule(l{3pt}r{3pt}){6-6} \cmidrule(l{3pt}r{3pt}){7-7}
\% object presence in input, 10-17 months & 68.61 & 14.13 & 55.90 & 15.72 & 46.46 & 16.85\\
N Input utterances, 10-17 months & 213.54 & 122.98 & 196.28 & 81.04 & 117.67 & 46.96\\
Productive Vocabulary 18m (CDI) & 58.89 & 60.76 & 64.10 & 61.97 & 13.00 & 9.49\\
\bottomrule
\end{tabular}
\end{table}

\begin{figure}
\centering
\includegraphics{SiblingsStudyText-anon-revisions_files/figure-latex/Figure-Speaker-count-1.pdf}
\caption{\label{fig:Figure-Speaker-count}Mean number of words produced by Mothers, Fathers and Siblings, as well as total family input (mother + father + sibling(s)), across sessions recorded between 10-17 months. Circles represent values for individual infants; red triangles show group means. In the case where the infant had two mothers, mean maternal input is shown.}
\end{figure}

Overall, for infants who had siblings, at least one other child was present in 68\% of recordings (n = 188 recordings, SD = 27\%). Wilcoxon Rank Sum tests comparing mean monthly input showed no difference between the amount of sibling input received by infants with one sibling compared with those with two or more siblings (\emph{W}=31, \emph{p}=.071, CI={[}-14.50,2{]}). Looking at mothers and fathers individually, infants with two or more siblings heard significantly less input from their mothers than those with one sibling (\emph{W}=5, \emph{p}\textless{} .001, CI={[}-124.88,-41.92{]}), while there was no difference between those with one vs.~no siblings (\emph{W}=125, \emph{p} = .985, CI={[}-48.23,51.50{]}). Finally, amount of paternal input did not differ between groups (one vs.~none: \emph{W}=102, \emph{p}=.393, CI={[}-10.21,61.30{]}; one vs.~2+: \emph{W}=21, \emph{p}=.945, CI={[}-74.33,56.54{]}).

\hypertarget{object-presence}{%
\subsubsection{Object presence}\label{object-presence}}

On average, 60\% of annotated utterances included a referent that was present and attended to by the infant (Mdn=0.61, SD=0.12). See Table \ref{tab:table-data-summary}. Consistent with our hypothesis that infants with more siblings would hear fewer words in referentially transparent conditions (i.e.~they would experience lower object presence) than those with fewer siblings, our models reveal a significant effect for sibling group on object presence (\(\chi^2 (2)\) = 27.33, \emph{p}\textless{} .001, \(R^2\)=0.55). See Table \ref{tab:table-input-model-summary} and Figure \ref{fig:Figure-object-presence}. Infants with no siblings experienced 22.14\% more object presence in their input than those with two or more siblings, and 12.71\% more than those with one sibling. Post-hoc comparisons revealed significant between-group differences: infants with no siblings experienced significantly more object presence than those with one sibling (\emph{W}=240, \emph{p}\textless{} .001, CI={[}0.07,0.20{]}; Bonferroni-corrected \emph{p}-threshold = .025). Likewise, infants with one sibling experienced significantly more object presence those with two or more siblings (\emph{W}=25, \emph{p}=.025, CI={[}-0.18,-0.01{]}).

\begin{figure}
\centering
\includegraphics{SiblingsStudyText-anon-revisions_files/figure-latex/Figure-object-presence-1.pdf}
\caption{\label{fig:Figure-object-presence}Percentage of input words produced with object presence across sibling groups. Error bars and black triangles show 95\% CIs and mean proportion of object presence across sibling groups. Dots indicate mean proportion of object presence per infant, collapsing across age and jittered horizontally for visual clarity.}
\end{figure}

\begin{longtable}[t]{ccccccc}
\caption{\label{tab:table-input-model-summary}Full model output from linear mixed effects regression models comparing our two input measures (object words produced in caregiver input and object presence) over time in relation to sibling group. Age in months was included as a fixed effect in both models, sex was included in the caregiver input model only; subject was included as a random effect.}\\
\toprule
Variable & Effect & Estimate & Std. Error & df & t value & p value\\
\midrule
Caregiver input & Intercept & 4.87 & 0.18 & 182.83 & 27.57 & <0.001\\
 & SibGroupOne & 0.01 & 0.15 & 43.00 & 0.04 & 0.965\\
 & SibGroup2+ & -0.49 & 0.17 & 43.00 & -2.94 & 0.005\\
 & month & 0.03 & 0.01 & 301.00 & 3.02 & 0.003\\
 & sexM & -0.18 & 0.13 & 43.00 & -1.38 & 0.173\\
\midrule
\addlinespace
Object presence & Intercept & 0.57 & 0.04 & 320.78 & 12.72 & <0.001\\
 & SibGroupOne & -0.13 & 0.03 & 43.00 & -3.80 & <0.001\\
 & SibGroup2+ & -0.22 & 0.04 & 43.00 & -5.87 & <0.001\\
 & month & 0.01 & 0.00 & 301.00 & 2.98 & 0.003\\
\bottomrule
\end{longtable}

\hypertarget{sibling-presence}{%
\subsubsection{Sibling presence}\label{sibling-presence}}

So far, our analysis takes into account the differences in input based on whether or not the target child has a sibling, but does not directly consider whether sibling presence in the recordings affected these variables. That is, if it is the active presence of the sibling that affects how the caretaker interacts with the target child, then we would expect to see a difference in our input measures when the sibling is present vs.~absent. On the other hand, if the very fact of having a sibling changes the way that a caregiver interacts with the infant regardless of whether any sibling is actual present on the scene, then no difference should be observed. While sibling presence in each recordings was not coded directly in the dataset, for this exploratory analysis we can get at this with an admittedly imperfect proxy: did the sibling produce nouns in the recording. If yes, we can safely assume they are present; if not we (less safely, but reasonably for initial exploratory purposes) assume they are not. As reported above, by this measure, at least one sibling was present in 68\% of recordings for the infants who had a sibling.

\begin{table}[!h]

\caption{\label{tab:sib-presence-table-data-summary}Data summary of our two input measures according to presence or absence of siblings during the recording.}
\centering
\begin{tabular}[t]{cccccc}
\toprule
\multicolumn{2}{c}{ } & \multicolumn{2}{c}{1 sibling} & \multicolumn{2}{c}{2+ siblings} \\
\cmidrule(l{3pt}r{3pt}){3-4} \cmidrule(l{3pt}r{3pt}){5-6}
\multicolumn{1}{c}{Variable} & \multicolumn{1}{c}{Sibling presence} & \multicolumn{1}{c}{Mean} & \multicolumn{1}{c}{SD} & \multicolumn{1}{c}{Mean} & \multicolumn{1}{c}{SD} \\
\cmidrule(l{3pt}r{3pt}){1-1} \cmidrule(l{3pt}r{3pt}){2-2} \cmidrule(l{3pt}r{3pt}){3-3} \cmidrule(l{3pt}r{3pt}){4-4} \cmidrule(l{3pt}r{3pt}){5-5} \cmidrule(l{3pt}r{3pt}){6-6}
N Input utterances, 10-17 months & Sibling not present & 136.60 & 107.68 & 84.58 & 58.43\\
 & Sibling present & 124.41 & 93.55 & 73.51 & 48.34\\
\% object presence in input, 10-17 months & Sibling not present & 68.81 & 12.36 & 59.22 & 14.25\\
 & Sibling present & 52.63 & 14.82 & 40.85 & 14.81\\
\bottomrule
\end{tabular}
\end{table}

\begin{figure}
\centering
\includegraphics{SiblingsStudyText-anon-revisions_files/figure-latex/figure-sibpresence-1.pdf}
\caption{\label{fig:figure-sibpresence}Difference in number of input words and \% of object presence in infants' input according to whether or not a sibling or siblings were present during the time of recording. Infants with no siblings were not included in the plots for visual ease. White shapes represent individual infants in the data; filled shapes represent means and 95\%CIs. For both plots, colours represent presence or absence of siblings during the recording session.}
\end{figure}

Descriptively, the presence of a sibling affected the amount of object presence in the data, but not the amount of input. See Table \ref{tab:sib-presence-table-data-summary} and Figure \ref{fig:figure-sibpresence}. In both cases, there was a gradient decline according to sibling group: infants with 2+ siblings heard fewer input words and less object presence than infants with one sibling, and this was more notable when siblings were present.

We ran Wilcoxon tests to determine the extent of these differences in the data. Note that in some cases, individual infants were recorded both with and without their sibling(s) across different sessions, meaning that the data takes into account mean `sibling present' and `sibling not present' values for each individual infant. First, looking at amount of input words, there was no statistical difference in the amount of input words produced by the caregivers when siblings were present vs.~when they weren't, for either sibling group (One: \emph{W}=17, \emph{p}=.396, CI={[}-102.17,149.75{]}; 2+: \emph{W}=26, \emph{p}=.105, CI={[}-7.71,31.37{]}). However, when siblings were present, infants with one sibling heard more input words than infants with two or more siblings (\emph{W}=24, \emph{p}=.011, CI={[}-64.23,-10.90{]}) (though note that the exact number of siblings present is not taken into account here). When siblings were not present, there were no group differences (\emph{W}=8, \emph{p}=.586, CI={[}-156.08,36.72{]}).

Next, comparing object presence within groups, we found that amount of object presence was significantly higher when a sibling was not present, across both groups (One: \emph{W}=98, \emph{p}=.007, CI={[}0.07,0.28{]}; 2+: \emph{W}=72, \emph{p}\textless{} .001, CI={[}0.11,0.37{]}). Comparing between groups, object presence was higher for infants with one versus 2 or more siblings even when a sibling was present (\emph{W}=15, \emph{p}=.003, CI={[}-0.23,-0.07{]}), but there was no difference between sibling groups when siblings were not present (\emph{W}=26, \emph{p}=.370, CI={[}-0.20,0.09{]}). Note that these effects should be interpreted with caution given the exploratory nature of this analysis.

\hypertarget{discussion}{%
\section{Discussion}\label{discussion}}

We investigated the nature of infant language development in relation to number of children in the household. Previous research found a delay in lexical acquisition for later-born infants (Fenson et al., 1994; Hoff, 2006), with differences in input across birth order reported as a root cause. Our results add several new dimensions to this, by testing for differences across more vs.~fewer older siblings, and by looking at input during child-centered home recordings. Infants with more siblings were reported to say fewer words by 18 months, heard fewer nouns from their parents, and were less likely to be attending to an object when hearing its label.

Importantly, and in contrast with some previous research (Hoff-Ginsberg, 1998; Oshima-Takane \& Robbins, 2003), infants with one sibling showed no decrement in lexical production and minimal reduction in input in comparison to first-born infants. That is, our results suggest that simply having a sibling does not contribute to input or vocabulary differences across children (as measured here), while having more than one siblings seems to do so. Indeed, infants with zero and one sibling had similar results for productive vocabulary, and parental noun input overall (though not object presence), and parental input was not affected by the presence or absence of the sibling in the room. In contrast, infants with two or more siblings said fewer words, and also heard fewer input words overall. They also had proportionally less object co-presence, compared with their peers, which was less likely to occur when the sibling was present than when they were not.

\hypertarget{the-sibling-effect}{%
\subsubsection{The sibling effect}\label{the-sibling-effect}}

When we considered the effect of sibling status -- that is, whether or not infants had any siblings, disregarding specific sibling number -- our findings showed that having siblings made no difference to infants' lexical production capacities. This contrasts with Hoff-Ginsberg (1998), who found that, by 18 months, laterborns exhibit lower language skills. However, Oshima-Takane and colleagues (1996) found no overall differences between first- and second-born children across a range of language measures taken at 21 months. Our results suggest that considering \emph{sibling quantity} may be a more sensitive way to reveal demographic effects than their (coarser-grained) first- vs.~later-born status. We find that the more older siblings a child had, the lower their reported productive vocabulary at 18 months. This adds to findings from Fenson and colleagues (1994), who found a weak but significant negative correlation between birth order and production of both words and gestures. Controlling for age, our model showed that for each additional older sibling, infants produced more than 30\% fewer words by 18 months.

While infants with more siblings heard less input speech overall, having one sibling did not significantly reduce the number of nouns in an infant's input. This is in direct contrast with reports from the literature; Hoff (2006) states that ``when a sibling is present, each child receives less speech directed solely at\ldots her \emph{because mothers produce the same amount of speech whether interacting with one or two children}'' (p.67, italics added). While this does not appear to be the case in the present dataset, it may be due to the circumstances of the home-recorded data: while siblings were present in many of the recordings (68\% of recordings in which the target child had a sibling), given the focus of the data collection, parents may have had a tendency to direct their attention - and consequently their linguistic input - more towards the target child; our samples also differed in other ways (e.g.~sociocultural context) that may have influenced the results as well. Alternatively, our results may diverge from those of Hoff (2006) due to the nature of our input measure, which only took nouns into account. That said, we find this alternative explanation unlikely given work by Bulgarelli and Bergelson (2020) showing that nouns are a reliable proxy for overall input in this dataset, suggesting that this measure provides an appropriate representation of overall input directed at the target child.

In contrast to the other results, our analysis of object presence showed a more linear `sibling effect'. In this case, even having one sibling led to fewer word-object pairs presented in the input. This was true regardless of whether or not other siblings were present, but object presence was further negatively affected by the presence of a sibling in the room. Presence of a labeled object with congruent input speech has been found to support early word learning across several studies. For instance, Bergelson and Aslin (2017) combined analysis of this home-recorded data at six months with an experimental study to show that word-object co-presence in naturalistic caregiver input correlated with comprehension of nouns (tested using eye-tracking). Relatedly, Gogate and colleagues (2000) propose that contingent word production supports the learning of novel word-object combinations, with ``multimodal motherese'' - whereby a target object word is produced in movement or touch-based synchrony with its referent - supporting word learning. More broadly, lower rates of referential transparency for common non-nouns like \emph{hi} and \emph{uh-oh} have been proposed to potentially explain why these words are learned later than common concrete nouns (Bergelson \& Swingley, 2013). While the present results on object presence don't speak directly to word learning, they do suggest that this potentially helpful learning support is less available for children with more siblings.

\hypertarget{siblingese-as-a-learning-opportunity}{%
\subsubsection{Siblingese as a learning opportunity?}\label{siblingese-as-a-learning-opportunity}}

We also found that infants with siblings did not hear much speech from their older brothers and sisters. Similar findings are reported in a lab-based interaction study by Oshima-Takane and Robbins (2003), who found that older siblings rarely talked directly to the target child; instead, most input from siblings was overheard speech from sibling-mother interactions. One possibility raised by these results is that perhaps parents are able to compensate or provide relatively similar input and learning support for one or two children, but once children outnumber parents, this balancing act of attention, care, and time becomes unwieldy. While the current sample is relatively limited and homogeneous in the family structures and demographics it includes, future work could fruitfully investigate this possibility by considering whether (controlling for other potential contributors like SES, Hoff-Ginsberg, 1998) the presence of more caregivers (whether parents, relatives, or other adults) helps foster language development.

Alternatively, second-borns might `even out' with children with no siblings due to a trade-off between direct attention from the caregiver and the possibility of more sophisticated social-communicative interactions. For these infants there is still ample opportunity to engage with the mother in one-to-one interactions, allowing a higher share of her attention than is available to third- or later-borns. Furthermore, triadic interactions can benefit the development of a number of linguistic and communication skills (Barton \& Tomasello, 1991; Dunn \& Shatz, 1989). Second-borns may also benefit from overheard speech in their input, supporting the acquisition of nouns and even more complex lexical categories (Floor \& Akhtar, 2006; Oshima-Takane et al., 1996). For infants with one sibling, the benefits of observing/overhearing interactions between sibling and caregiver, as well as the possibility for partaking in such interactions, may outweigh the decrease in some aspects of the input (i.e., in our data, only observed in object presence). Having more than one sibling may throw this off-balance, such that the possibilities for both supportive one-to-one input \emph{and} more sophisticated interactions are simultaneously diminished.

Importantly, the present results make no claims about eventual outcomes for these children: generally speaking, regardless of sibling number, all typically-developing infants reach full and fluent language use. Indeed, some research suggests that sibling effects, while they may be clear in early development, are not always sustained into childhood; e.g.~twins are known to experience a delay in language development into the third year, but are quick to catch up thereafter (Dales, 1969; Tomasello, Mannle, \& Kruger, 1986). This demonstrates the cognitive adaptability of early development, which brings about the acquisition of language across varying and allegedly `imperfect' learning environments. Infants' capacity to develop linguistic skills from the resources that are available to them -- whether that is infant-directed object labels or overheard abstract concepts -- highlights the dynamic and adaptable nature of early cognitive development, and a system that is sufficiently robust to bring about the same outcome across populations.

\hypertarget{limitations}{%
\subsubsection{Limitations}\label{limitations}}

Of course, the `success' of early language development is defined by how success is measured. Here we chose word production as our measure of linguistic capability; we did not consider other equally valid measures such as language comprehension or early social-interaction skills. Similarly, our input measures focused on nouns; other lexical classes may reveal different effects, though they are generally far sparser in production until toddlerhood. Our analysis of productive and receptive vocabulary relied on parental report data; as our two anonymous reviewers both suggested, this method could have biased our first-born sample towards more accurate or larger vocabulary reports owing to their parents having more time and attention to spend observing their vocabulary development (see Kartushina et al., 2022 for a discussion of this possibility in light of the COVID-19 pandemic). However, we were able to validate this measure by running correlation tests between reported (CDI) vocabulary and the number of words produced by each infant in both the audio and video data. Across both type and token counts, reported productive vocabulary size correlated strongly with words produced in the audio/video data (rho values all between 0.54-0.60).

There is also some imbalance in group sizes across our data; our sample was not pre-selected for sibling number, and so group sizes are unmatched across the analysis. Including a larger number of infants with 2+ siblings may have revealed a different pattern of results. Finally, more work across wider and larger populations is necessary to unpack the generalizability of the present results. Our sample is reflective of average household sizes in middle-class families across North America and Western Europe (Office for National Statistics, 2018; United States Census Bureau, 2010), but it is not unusual in some communities and parts of the world for households to include between three and six children on average (Institute for Family Studies \& Wheatley Institution, 2019). Adding to this, it is also necessary to consider cross-cultural differences in the way children are addressed by their parents, other caretakers, and other children (Bunce et al., 2020; Casillas, Brown, \& Levinson, 2019; Shneidman \& Goldin‐Meadow, 2012). For instance, Bunce and colleagues (2020) find relatively similar rates of target child directed speech across US, Canadian, Argentinian, UK, Papuan and Mayan samples, some differences in who the input comes from, and large effects of number of talkers present. These results suggest that caution is advisable before generalizing the current results to any other socio-cultural contexts, but also pose exciting open questions regarding what variability in experiences do -- or don't -- change about early language interaction and development.

\hypertarget{conclusion}{%
\subsection{Conclusion}\label{conclusion}}

Our results with English-learning infants in the US support prior findings from the literature showing that later-born infants have slower lexical acquisition than their first-born peers. However, we highlight an important difference from previous findings, namely that in the present sample, second-born infants show no such effect, while infants with more than two siblings have significantly smaller productive vocabularies at age 18 months. Correspondingly, we identified parallel group differences in overall noun input and object presence. While we did not test these corresponding measures directly, our results suggest that having more siblings affects a child's early language environment, which in turn may lead to slower vocabulary growth in the first 18 months of life. We look forward to future studies considering the granularity of more versus fewer siblings, and how this relates to language abilities over the course of development.

\newpage

\hypertarget{references}{%
\section{References}\label{references}}

\begingroup
\setlength{\parindent}{-0.5in}
\setlength{\leftskip}{0.5in}

\hypertarget{refs}{}
\begin{CSLReferences}{1}{0}
\leavevmode\vadjust pre{\hypertarget{ref-akhtar_robustness_2005}{}}%
Akhtar, N. (2005). The robustness of learning through overhearing. \emph{Developmental Science}, \emph{8}(2), 199--209. \url{https://doi.org/10.1111/j.1467-7687.2005.00406.x}

\leavevmode\vadjust pre{\hypertarget{ref-akhtar_learning_2001}{}}%
Akhtar, N., Jipson, J., \& Callanan, M. A. (2001). Learning {Words} {Through} {Overhearing}. \emph{Child Development}, \emph{72}(2), 416--430. Retrieved from \url{http://www.jstor.org/stable/1132404}

\leavevmode\vadjust pre{\hypertarget{ref-anderson_linking_2021}{}}%
Anderson, N. J., Graham, S. A., Prime, H., Jenkins, J. M., \& Madigan, S. (2021). Linking {Quality} and {Quantity} of {Parental} {Linguistic} {Input} to {Child} {Language} {Skills}: {A} {Meta}-{Analysis}. \emph{Child Development}, \emph{92}(2), 484--501. \url{https://doi.org/10.1111/cdev.13508}

\leavevmode\vadjust pre{\hypertarget{ref-barnes_characteristics_1983}{}}%
Barnes, S., Gutfreund, M., Satterly, D., \& Wells, G. (1983). Characteristics of adult speech which predict children's language development. \emph{Journal of Child Language}, \emph{10}(1), 65--84. \url{https://doi.org/10.1017/S0305000900005146}

\leavevmode\vadjust pre{\hypertarget{ref-barton_joint_1991}{}}%
Barton, M. E., \& Tomasello, M. (1991). \emph{Joint {Attention} and {Conversation} in {Mother}-{Infant}-{Sibling} {Triads}}. \emph{62}(3), 517--529.

\leavevmode\vadjust pre{\hypertarget{ref-bergelson_day_2019}{}}%
Bergelson, E., Amatuni, A., Dailey, S., Koorathota, S., \& Tor, S. (2019). Day by day, hour by hour: {Naturalistic} language input to infants. \emph{Developmental Science}, \emph{22}(1), e12715. \url{https://doi.org/10.1111/desc.12715}

\leavevmode\vadjust pre{\hypertarget{ref-bergelson_nature_2017}{}}%
Bergelson, E., \& Aslin, R. N. (2017). Nature and origins of the lexicon in 6-mo-olds. \emph{Proceedings of the National Academy of Sciences}, \emph{114}(49), 12916--12921. \url{https://doi.org/10.1073/pnas.1712966114}

\leavevmode\vadjust pre{\hypertarget{ref-bergelson_acquisition_2013}{}}%
Bergelson, E., \& Swingley, D. (2013). The {Acquisition} of {Abstract} {Words} by {Young} {Infants}. \emph{Cognition}, \emph{127}(3), 391--397. \url{https://doi.org/10.1038/jid.2014.371}

\leavevmode\vadjust pre{\hypertarget{ref-berglund_communicative_2005}{}}%
Berglund, E., Eriksson, M., \& Westerlund, M. (2005). Communicative skills in relation to gender, birth order, childcare and socioeconomic status in 18-month-old children. \emph{Scandinavian Journal of Psychology}, \emph{46}(6), 485--491. \url{https://doi.org/10.1111/j.1467-9450.2005.00480.x}

\leavevmode\vadjust pre{\hypertarget{ref-braginsky_consistency_2019}{}}%
Braginsky, M., Yurovsky, D., Marchman, V. A., \& Frank, M. C. (2019). Consistency and {Variability} in {Children}'s {Word} {Learning} {Across} {Languages}. \emph{Open Mind}, \emph{3}, 52--67. \url{https://doi.org/10.1162/opmi_a_00026}

\leavevmode\vadjust pre{\hypertarget{ref-bulgarelli_look_2020}{}}%
Bulgarelli, F., \& Bergelson, E. (2020). Look who's talking: {A} comparison of automated and human-generated speaker tags in naturalistic day-long recordings. \emph{Behavior Research Methods}, \emph{52}(2), 641--653. \url{https://doi.org/10.3758/s13428-019-01265-7}

\leavevmode\vadjust pre{\hypertarget{ref-bunce_cross-cultural_2020}{}}%
Bunce, J., Soderstrom, M., Bergelson, E., Rosemberg, C., Stein, A., Migdalek, M., \& Casillas, M. (2020). \emph{A cross-cultural examination of young children's everyday language experiences} {[}Preprint{]}.

\leavevmode\vadjust pre{\hypertarget{ref-cartmill_quality_2013}{}}%
Cartmill, E. a., Armstrong, B. F., Gleitman, L. R., Goldin-Meadow, S., Medina, T. N., \& Trueswell, J. C. (2013). Quality of early parent input predicts child vocabulary 3 years later. \emph{Proceedings of the National Academy of Sciences of the United States of America}. \url{https://doi.org/10.1073/pnas.1309518110}

\leavevmode\vadjust pre{\hypertarget{ref-casillas_early_2019}{}}%
Casillas, M., Brown, P., \& Levinson, S. C. (2019). Early {Language} {Experience} in a {Tseltal} {Mayan} {Village}. \emph{Child Development}, \emph{EarlyView article}. \url{https://doi.org/10.1111/cdev.13349}

\leavevmode\vadjust pre{\hypertarget{ref-dales_motor_1969}{}}%
Dales, R. J. (1969). Motor and language development of twins during the first three years. \emph{The Journal of Genetic Psychology; Provincetown, Mass., Etc.}, \emph{114}(2), 263--271. Retrieved from \url{https://search.proquest.com/docview/1297124434/citation/D928716F9A7E4AEFPQ/1}

\leavevmode\vadjust pre{\hypertarget{ref-dunn_becoming_1989}{}}%
Dunn, J., \& Shatz, M. (1989). Becoming a {Conversationalist} despite ({Or} {Because} of) {Having} an {Older} {Sibling}. \emph{Child Development}, \emph{60}(2), 399--410.

\leavevmode\vadjust pre{\hypertarget{ref-fenson_variability_1994}{}}%
Fenson, L., Dale, P. S., Reznick, J. S., Bates, E., Thal, D. J., Pethick, M., Stephen J. Tomasello, \ldots{} Stiles, J. (1994). Variability in {Early} {Communicative} {Development}. \emph{Monographs of the Society for Research in Child Development}, \emph{59}. \url{https://doi.org/10.2307/1166093}

\leavevmode\vadjust pre{\hypertarget{ref-fitch_toddlers_2020}{}}%
Fitch, A., Liberman, A. M., Luyster, R., \& Arunachalam, S. (2020). Toddlers' word learning through overhearing: {Others}' attention matters. \emph{Journal of Experimental Child Psychology}, \emph{193}, 104793. \url{https://doi.org/10.1016/j.jecp.2019.104793}

\leavevmode\vadjust pre{\hypertarget{ref-floor_can_2006}{}}%
Floor, P., \& Akhtar, N. (2006). Can 18-month-old infants learn words by listening in on conversations? \emph{Infancy}, \emph{9}(3), 327--339. \url{https://doi.org/10.1207/s15327078in0903_4}

\leavevmode\vadjust pre{\hypertarget{ref-foushee_self-directed_2021}{}}%
Foushee, R., Srinivasan, M., \& Xu, F. (2021). Self-directed learning by preschoolers in a naturalistic overhearing context. \emph{Cognition}, \emph{206}, 104415. \url{https://doi.org/10.1016/j.cognition.2020.104415}

\leavevmode\vadjust pre{\hypertarget{ref-frank_variability_2021}{}}%
Frank, M. C., Braginsky, M., Yurovsky, D., \& Marchman, V. A. (2021). \emph{Variability and {Consistency} in {Early} {Language} {Learning}: {The} {Wordbank} {Project}}. MIT Press.

\leavevmode\vadjust pre{\hypertarget{ref-gogate_study_2000}{}}%
Gogate, L. J., Bahrick, L. E., \& Watson, J. D. (2000). A {Study} of {Multimodal} {Motherese}: {The} {Role} of {Temporal} {Synchrony} between {Verbal} {Labels} and {Gestures}. \emph{Child Development}, \emph{71}(4), 878--894. \url{https://doi.org/10.1111/1467-8624.00197}

\leavevmode\vadjust pre{\hypertarget{ref-gurgand_influence_2022}{}}%
Gurgand, L., Lamarque, L., Havron, N., Bernard, J. Y., Ramus, F., \& Peyre, H. (2022). The influence of sibship composition on language development at 2 years of age in the {ELFE} birth cohort study. \emph{Developmental Science}, \emph{n/a}(n/a), e13356. \url{https://doi.org/10.1111/desc.13356}

\leavevmode\vadjust pre{\hypertarget{ref-havron_effect_2022}{}}%
Havron, N., Lovcevic, I., Kee, M., Chen, H., Chong, Y., Daniel, M., \ldots{} Tsuji, S. (2022). The {Effect} of {Older} {Sibling}, {Postnatal} {Maternal} {Stress}, and {Household} {Factors} on {Language} {Development} in {Two}- to {Four}-{Year}-{Old} {Children}. \emph{Developmental Psychology}, \emph{58}(11), 2096--2113. \url{https://doi.org/10.31234/osf.io/m9w48}

\leavevmode\vadjust pre{\hypertarget{ref-havron_effect_2019-1}{}}%
Havron, N., Ramus, F., Heude, B., Forhan, A., Cristia, A., Peyre, H., \ldots{} Thiebaugeorges, O. (2019). The {Effect} of {Older} {Siblings} on {Language} {Development} as a {Function} of {Age} {Difference} and {Sex}. \emph{Psychological Science}, \emph{30}(9), 1333--1343. \url{https://doi.org/10.1177/0956797619861436}

\leavevmode\vadjust pre{\hypertarget{ref-hoff_how_2006}{}}%
Hoff, E. (2006). How social contexts support and shape language development. \emph{Developmental Review}, \emph{26}(1), 55--88. \url{https://doi.org/10.1016/j.dr.2005.11.002}

\leavevmode\vadjust pre{\hypertarget{ref-hoff-ginsberg_relation_1998}{}}%
Hoff-Ginsberg, E. (1998). The relation of birth order and socioeconomic status to children's language experience and language development. \emph{Applied Psycholinguistics}, \emph{19}(4), 603--629. \url{https://doi.org/10.1017/S0142716400010389}

\leavevmode\vadjust pre{\hypertarget{ref-institute_for_family_studies_world_2019}{}}%
Institute for Family Studies, \& Wheatley Institution. (2019). \emph{World family map 2019: {Mapping} family change and child well-being outcomes}. Charlottesville, VA: Institute for Family Studies. Retrieved from \url{https://ifstudies.org/reports/world-family-map/2019/executive-summary}

\leavevmode\vadjust pre{\hypertarget{ref-jakiela_big_2020}{}}%
Jakiela, P., Ozier, O., Fernald, L., \& Knauer, H. (2020). Big sisters. \emph{Policy Research Working Papers}, \emph{9454}. Retrieved from \url{https://openknowledge.worldbank.org/server/api/core/bitstreams/54b5d934-3263-5e51-93af-7d1e9e619804/content}

\leavevmode\vadjust pre{\hypertarget{ref-jones_language_1987}{}}%
Jones, C. P., \& Adamson, L. B. (1987). \emph{Language {Use} in {Mother}-{Child} and {Mother}-{Child}-{Sibling} {Interactions}}. \emph{58}(2), 356--366.

\leavevmode\vadjust pre{\hypertarget{ref-kartushina_covid-19_2022}{}}%
Kartushina, N., Mani, N., Aktan-Erciyes, A., Alaslani, K., Aldrich, N. J., Almohammadi, A., \ldots{} Mayor, J. (2022). {COVID}-19 first lockdown as a window into language acquisition: Associations between caregiver-child activities and vocabulary gains. \emph{Language Development Research}, \emph{2}(1). \url{https://doi.org/10.34842/abym-xv34}

\leavevmode\vadjust pre{\hypertarget{ref-karwath_sibling_2014}{}}%
Karwath, C., Relikowski, I., \& Schmitt, M. (2014). Sibling structure and educational achievement: How do the number of siblings, birth order, and birth spacing affect children's vocabulary competences? \emph{Journal of Family Research}, \emph{26}(3), 372--396. \url{https://doi.org/10.3224/zff.v26i3.18993}

\leavevmode\vadjust pre{\hypertarget{ref-kuznetsova_lmertest_2017}{}}%
Kuznetsova, A., Brockhoff, P. B., \& Christensen, R. H. B. (2017). \{{lmerTest}\} {Package}: {Tests} in {Linear} {Mixed} {Effects} {Models}. \emph{Journal of Statistical Software}, \emph{82}(13), 1--26. \url{https://doi.org/10.18637/jss.v082.i13}

\leavevmode\vadjust pre{\hypertarget{ref-laing_mothers_2019}{}}%
Laing, C. E., \& Bergelson, E. (2019). Mothers' {Work} {Status} and 17-{Month}-{Olds}' {Productive} {Vocabulary}. \emph{Infancy}, \emph{24}(1), 101--109. \url{https://doi.org/10.1111/infa.12265}

\leavevmode\vadjust pre{\hypertarget{ref-moore_point_2019}{}}%
Moore, C., Dailey, S., Garrison, H., Amatuni, A., \& Bergelson, E. (2019). Point, {Walk}, {Talk}: {Links} {Between} {Three} {Early} {Milestones}, {From} {Observation} and {Parental} {Report}. \emph{Developmental Psychology}. \url{https://doi.org/10.1037/dev0000738}

\leavevmode\vadjust pre{\hypertarget{ref-office_for_national_statistics_families_2018}{}}%
Office for National Statistics. (2018). \emph{Families with dependent children by number of children, {UK}, 1996 to 2017} (No. 008855). Office for National Statistics. Retrieved from Office for National Statistics website: \url{https://www.ons.gov.uk/peoplepopulationandcommunity/birthsdeathsandmarriages/families/adhocs/008855familieswithdependentchildrenbynumberofchildrenuk1996to2017}

\leavevmode\vadjust pre{\hypertarget{ref-oshima-takane_birth_1996}{}}%
Oshima-Takane, Y., Goodz, E., \& Derevensky, J. L. (1996). \emph{Birth {Order} {Effects} on {Early} {Language} {Development} : {Do} {Secondborn} {Children} {Learn} from {Overheard} {Speech} ? {Author} ( s ): {Yuriko} {Oshima}-{Takane} , {Elizabeth} {Goodz} and {Jeffrey} {L} . {Derevensky} {Published} by : {Wiley} on behalf of the {Society} for {Research} in {Child} {De}}. \emph{67}(2), 621--634.

\leavevmode\vadjust pre{\hypertarget{ref-oshima-takane_linguistic_2003}{}}%
Oshima-Takane, Y., \& Robbins, M. (2003). Linguistic environment of secondborn children. \emph{First Language}, \emph{23}(1), 21--40. https://doi.org/\url{http://dx.doi.org/10.1177/0142723703023001002}

\leavevmode\vadjust pre{\hypertarget{ref-peyre_differential_2016}{}}%
Peyre, H., Bernard, J. Y., Hoertel, N., Forhan, A., Charles, M.-A., De Agostini, M., \ldots{} Ramus, F. (2016). Differential effects of factors influencing cognitive development at the age of 5-to-6 years. \emph{Cognitive Development}, \emph{40}, 152--162. \url{https://doi.org/10.1016/j.cogdev.2016.10.001}

\leavevmode\vadjust pre{\hypertarget{ref-pine_variation_1995}{}}%
Pine, J. M. (1995). Variation in {Vocabulary} {Development} as a {Function} of {Birth} {Order}. \emph{Child Development}, \emph{66}(1), 272--281.

\leavevmode\vadjust pre{\hypertarget{ref-r_core_team_r_2019}{}}%
R Core Team. (2019). \emph{R: {A} {Language} {Environment} for {Statistical} {Computing}}. R Foundation for Statistical Computing. Retrieved from \url{https://www.R-project.org/}

\leavevmode\vadjust pre{\hypertarget{ref-ramirez-esparza_look_2014}{}}%
Ramírez-Esparza, N., García-Sierra, A., \& Kuhl, P. K. (2014). Look who's talking: Speech style and social context in language input to infants are linked to concurrent and future speech development. \emph{Developmental Science}, \emph{17}(6), 880--891. \url{https://doi.org/10.1016/j.surg.2006.10.010.Use}

\leavevmode\vadjust pre{\hypertarget{ref-rowe_analyzing_2020}{}}%
Rowe, M. L., \& Snow, C. E. (2020). Analyzing input quality along three dimensions: Interactive, linguistic, and conceptual. \emph{Journal of Child Language}, \emph{47}(1), 5--21. \url{https://doi.org/10.1017/S0305000919000655}

\leavevmode\vadjust pre{\hypertarget{ref-shneidman_language_2012}{}}%
Shneidman, L. A., \& Goldin‐Meadow, S. (2012). Language input and acquisition in a {Mayan} village: How important is directed speech? \emph{Developmental Science}, \emph{15}(5), 659--673. \url{https://doi.org/10.1111/j.1467-7687.2012.01168.x}

\leavevmode\vadjust pre{\hypertarget{ref-soderstrom_beyond_2007}{}}%
Soderstrom, M. (2007). Beyond babytalk: {Re}-evaluating the nature and content of speech input to preverbal infants. \emph{Developmental Review}, \emph{27}(4), 501--532. \url{https://doi.org/10.1016/j.dr.2007.06.002}

\leavevmode\vadjust pre{\hypertarget{ref-stern_prosody_1983}{}}%
Stern, D. N., Spieker, S., Barnett, R. K., \& MacKain, K. (1983). The {Prosody} of {Maternal} {Speech}: {Infant} {Age} and {Context} {Related} {Changes}. \emph{Journal of Child Language}, \emph{10}(1), 1--15. \url{https://doi.org/10.1017/S0305000900005092}

\leavevmode\vadjust pre{\hypertarget{ref-tomasello_linguistic_1986}{}}%
Tomasello, M., Mannle, S., \& Kruger, A. C. (1986). Linguistic environment of 1- to 2-year-old twins. \emph{Developmental Psychology}, \emph{22}(2), 169--176. \url{https://doi.org/10.1037/0012-1649.22.2.169}

\leavevmode\vadjust pre{\hypertarget{ref-united_states_census_bureau_household_2010}{}}%
United States Census Bureau. (2010). \emph{Household {Type} by {Number} of {People} {Under} 18 {Years}} (No. PCT16). Retrieved from \url{https://data.census.gov/cedsci/table?q=number\%20of\%20children\&hidePreview=false\&tid=DECENNIALSF12010.PCT16\&t=Children\&vintage=2018}

\leavevmode\vadjust pre{\hypertarget{ref-zambrana_impact_2012}{}}%
Zambrana, I. M., Ystrom, E., \& Pons, F. (2012). Impact of {Gender}, {Maternal} {Education}, and {Birth} {Order} on the {Development} of {Language} {Comprehension}: {A} {Longitudinal} {Study} from 18 to 36 {Months} of {Age}. \emph{Journal of Developmental \& Behavioral Pediatrics}, \emph{33}(2), 146--155. \url{https://doi.org/10.1097/DBP.0b013e31823d4f83}

\end{CSLReferences}

\endgroup


\end{document}
