% Options for packages loaded elsewhere
\PassOptionsToPackage{unicode}{hyperref}
\PassOptionsToPackage{hyphens}{url}
%
\documentclass[
  man,floatsintext]{apa6}
\usepackage{amsmath,amssymb}
\usepackage{lmodern}
\usepackage{iftex}
\ifPDFTeX
  \usepackage[T1]{fontenc}
  \usepackage[utf8]{inputenc}
  \usepackage{textcomp} % provide euro and other symbols
\else % if luatex or xetex
  \usepackage{unicode-math}
  \defaultfontfeatures{Scale=MatchLowercase}
  \defaultfontfeatures[\rmfamily]{Ligatures=TeX,Scale=1}
\fi
% Use upquote if available, for straight quotes in verbatim environments
\IfFileExists{upquote.sty}{\usepackage{upquote}}{}
\IfFileExists{microtype.sty}{% use microtype if available
  \usepackage[]{microtype}
  \UseMicrotypeSet[protrusion]{basicmath} % disable protrusion for tt fonts
}{}
\makeatletter
\@ifundefined{KOMAClassName}{% if non-KOMA class
  \IfFileExists{parskip.sty}{%
    \usepackage{parskip}
  }{% else
    \setlength{\parindent}{0pt}
    \setlength{\parskip}{6pt plus 2pt minus 1pt}}
}{% if KOMA class
  \KOMAoptions{parskip=half}}
\makeatother
\usepackage{xcolor}
\usepackage{graphicx}
\makeatletter
\def\maxwidth{\ifdim\Gin@nat@width>\linewidth\linewidth\else\Gin@nat@width\fi}
\def\maxheight{\ifdim\Gin@nat@height>\textheight\textheight\else\Gin@nat@height\fi}
\makeatother
% Scale images if necessary, so that they will not overflow the page
% margins by default, and it is still possible to overwrite the defaults
% using explicit options in \includegraphics[width, height, ...]{}
\setkeys{Gin}{width=\maxwidth,height=\maxheight,keepaspectratio}
% Set default figure placement to htbp
\makeatletter
\def\fps@figure{htbp}
\makeatother
\setlength{\emergencystretch}{3em} % prevent overfull lines
\providecommand{\tightlist}{%
  \setlength{\itemsep}{0pt}\setlength{\parskip}{0pt}}
\setcounter{secnumdepth}{-\maxdimen} % remove section numbering
% Make \paragraph and \subparagraph free-standing
\ifx\paragraph\undefined\else
  \let\oldparagraph\paragraph
  \renewcommand{\paragraph}[1]{\oldparagraph{#1}\mbox{}}
\fi
\ifx\subparagraph\undefined\else
  \let\oldsubparagraph\subparagraph
  \renewcommand{\subparagraph}[1]{\oldsubparagraph{#1}\mbox{}}
\fi
\newlength{\cslhangindent}
\setlength{\cslhangindent}{1.5em}
\newlength{\csllabelwidth}
\setlength{\csllabelwidth}{3em}
\newlength{\cslentryspacingunit} % times entry-spacing
\setlength{\cslentryspacingunit}{\parskip}
\newenvironment{CSLReferences}[2] % #1 hanging-ident, #2 entry spacing
 {% don't indent paragraphs
  \setlength{\parindent}{0pt}
  % turn on hanging indent if param 1 is 1
  \ifodd #1
  \let\oldpar\par
  \def\par{\hangindent=\cslhangindent\oldpar}
  \fi
  % set entry spacing
  \setlength{\parskip}{#2\cslentryspacingunit}
 }%
 {}
\usepackage{calc}
\newcommand{\CSLBlock}[1]{#1\hfill\break}
\newcommand{\CSLLeftMargin}[1]{\parbox[t]{\csllabelwidth}{#1}}
\newcommand{\CSLRightInline}[1]{\parbox[t]{\linewidth - \csllabelwidth}{#1}\break}
\newcommand{\CSLIndent}[1]{\hspace{\cslhangindent}#1}
\ifLuaTeX
\usepackage[bidi=basic]{babel}
\else
\usepackage[bidi=default]{babel}
\fi
\babelprovide[main,import]{english}
% get rid of language-specific shorthands (see #6817):
\let\LanguageShortHands\languageshorthands
\def\languageshorthands#1{}
% Manuscript styling
\usepackage{upgreek}
\captionsetup{font=singlespacing,justification=justified}

% Table formatting
\usepackage{longtable}
\usepackage{lscape}
% \usepackage[counterclockwise]{rotating}   % Landscape page setup for large tables
\usepackage{multirow}		% Table styling
\usepackage{tabularx}		% Control Column width
\usepackage[flushleft]{threeparttable}	% Allows for three part tables with a specified notes section
\usepackage{threeparttablex}            % Lets threeparttable work with longtable

% Create new environments so endfloat can handle them
% \newenvironment{ltable}
%   {\begin{landscape}\centering\begin{threeparttable}}
%   {\end{threeparttable}\end{landscape}}
\newenvironment{lltable}{\begin{landscape}\centering\begin{ThreePartTable}}{\end{ThreePartTable}\end{landscape}}

% Enables adjusting longtable caption width to table width
% Solution found at http://golatex.de/longtable-mit-caption-so-breit-wie-die-tabelle-t15767.html
\makeatletter
\newcommand\LastLTentrywidth{1em}
\newlength\longtablewidth
\setlength{\longtablewidth}{1in}
\newcommand{\getlongtablewidth}{\begingroup \ifcsname LT@\roman{LT@tables}\endcsname \global\longtablewidth=0pt \renewcommand{\LT@entry}[2]{\global\advance\longtablewidth by ##2\relax\gdef\LastLTentrywidth{##2}}\@nameuse{LT@\roman{LT@tables}} \fi \endgroup}

% \setlength{\parindent}{0.5in}
% \setlength{\parskip}{0pt plus 0pt minus 0pt}

% \usepackage{etoolbox}
\makeatletter
\patchcmd{\HyOrg@maketitle}
  {\section{\normalfont\normalsize\abstractname}}
  {\section*{\normalfont\normalsize\abstractname}}
  {}{\typeout{Failed to patch abstract.}}
\patchcmd{\HyOrg@maketitle}
  {\section{\protect\normalfont{\@title}}}
  {\section*{\protect\normalfont{\@title}}}
  {}{\typeout{Failed to patch title.}}
\makeatother
\shorttitle{Effect of sibling number on language}
\keywords{Siblings, Lexical Development, Input Effects, Language Acquisition\newline\indent Word count: X}
\usepackage{lineno}

\linenumbers
\usepackage{csquotes}
\ifLuaTeX
  \usepackage{selnolig}  % disable illegal ligatures
\fi
\IfFileExists{bookmark.sty}{\usepackage{bookmark}}{\usepackage{hyperref}}
\IfFileExists{xurl.sty}{\usepackage{xurl}}{} % add URL line breaks if available
\urlstyle{same} % disable monospaced font for URLs
\hypersetup{
  pdftitle={Analysing the effect of sibling number on input and output in the first 18 months},
  pdfauthor={Catherine Laing1 \& Elika Bergelson2},
  pdflang={en-EN},
  pdfkeywords={Siblings, Lexical Development, Input Effects, Language Acquisition},
  hidelinks,
  pdfcreator={LaTeX via pandoc}}

\title{Analysing the effect of sibling number on input and output in the first 18 months}
\author{Catherine Laing\textsuperscript{1} \& Elika Bergelson\textsuperscript{2}}
\date{}


\affiliation{\vspace{0.5cm}\textsuperscript{1} University of York, York, UK\\\textsuperscript{2} Duke University, Durham, NC, USA}

\abstract{
The `sibling effect' has been widely reported in studies examining a breadth of topics in the academic literature, suggesting firstborn children are advantaged across a range of cognitive, educational and health-based measures compared with their later-born peers. Expanding on this literature using naturalistic home-recorded data and parental vocabulary report, we find that early language outcomes vary by number of siblings. Specifically, we find that children with two or more older siblings - but not one - had smaller vocabularies at 18 months, and heard less input from caregivers across several measures. We discuss implications regarding what infants experience and learn across a range of family sizes in infancy.

}



\begin{document}
\maketitle

Many studies assume a theoretical ``optimum'' environment for early language development, whereby the input is tailored to a single infant's needs, changing over time as language capacity develops (e.g. Soderstrom, 2007; Stern, Spieker, Barnett, \& MacKain, 1983). However, for many infants and for many reasons, language acquisition occurs amid various domestic and social factors that can influence the learning environment, including the presence of older siblings in the home (Fenson et al., 1994). According to the United States Census Bureau (2010), around one third of children are born into households with at least one other infant present, and one in every five infants is acquiring language in a household shared with two or more other children. Similar statistics are reported for British infants (Office for National Statistics, 2018), where the average household has 1.75 children, and 15\% of households have three children or more. In this paper, we consider the role of siblings in the early language environment of English-learning infants. We use naturalistic home-recorded data to measure input in earlier- and later-born infants in relation to their lexical development over the first 18 months of life.

Prior research suggests that infants born to households with older children may be slower to learn language. Fenson and colleagues (1994) found that by 30 months of age, children with older siblings performed worse than those with no siblings across measures of productive vocabulary, use of word combinations, and mean length of utterance (MLU). This `sibling effect' may be manifested in input differences between first- and later-born children: some research finds that infants with older siblings hear less speech aimed specifically at them, and what they do hear is understood to be linguistically less supportive of early language development (Hoff-Ginsberg, 1998; Oshima-Takane \& Robbins, 2003). Contrastingly, some studies have noted linguistic \emph{advantages} for later-borns, who may have stronger social-communicative skills (Hoff, 2006), better understanding of pronouns (Oshima-Takane, Goodz, \& Derevensky, 1996), and better conversational abilities (Dunn \& Shatz, 1989). Overall, while the particulars differ across studies, prior work suggests that the presence of siblings in the home leads to differences in infants' early linguistic experience.

Numerous studies have attempted to better understand how siblings affect the language development trajectory, with comparisons of language acquisition across first- and later-borns, and analyses of mothers' input in dyadic (infant + mother) and triadic (infant + mother + older sibling) situations. Findings are mixed, but overall two general conclusions can be drawn. First, analyses consistently show that infants with older siblings generally have slower \emph{vocabulary development} (Berglund, Eriksson, \& Westerlund, 2005; Fenson et al., 1994; Pine, 1995; Zambrana, Ystrom, \& Pons, 2012), but this is often marginal, and typical of the earliest stages of language learning. Hoff-Ginsberg (1998) shows first-borns to have better lexical and syntactic skills up until 2;5, but later-born infants had better conversational abilities during the same time-period. Some of these differences may relate to insufficient power to detect relatively small effects or simultaneous contributing factors. For instance, using a large longitudinal dataset of french-learning 2-5 year olds, Havron and colleagues (2019) find no effect of age gap between siblings but lower standardized language scores in children with older brothers (but not sisters) relative to those without siblings, based on parental report and direct battery assessments.

The second general finding pertains to sibling-related differences in the early linguistic \emph{environment}: infants with no siblings receive more input overall, and this more closely reflects what is typically considered to be `high quality' input in the extant literature. Indeed, the very presence of a sibling in the linguistic environment changes the way language is used. When siblings are present (i.e.~triadic interactions), mothers' input is more focused on regulating behaviour, as opposed to the language-focused speech that is common in dyadic contexts (Oshima-Takane \& Robbins, 2003). Reports show that MLU is longer in the input of first-born infants (Hoff-Ginsberg, 1998; but see also Oshima-Takane \& Robbins, 2003 for a comparison of dyadic and triadic contexts), who also hear more questions directed at them than later-borns. Both Jones and Adamson (1987) and Oshima-Takane and Robbins (2003) report no difference between the overall number of word types produced by mothers in dyadic and triadic settings, but the proportion of speech directed at the target infant is drastically reduced when input is shared with siblings.

As Hoff (2006) explains, infants with siblings have less experience of speech directed at them, but they do have an advantage over their first-born peers in that they are subject to more overheard speech. Indeed, the input of first-borns may be explicitly tailored to their needs, but equally this means it might be less varied. Barton and Tomasello (1991) show that by as early as 19 months, infants with siblings are already able to take part in triadic conversations, which were almost three times longer than dyadic conversations. The authors suggest that the presence of siblings may shift the learning context, and facilitate infants' participation in communicative interactions: infants are under less pressure to participate in a triadic interaction, meaning the conversation can continue even when the infant is unable to respond. As a result, the infants in Barton and Tomasello's study took more conversational turns in triadic interactions than dyadic ones.

There thus may be a trade-off in development between highly supportive one-to-one input from a caregiver (cf. Ramírez-Esparza, García-Sierra, \& Kuhl, 2014) and the potential benefits drawn from communicating with a sibling. In the present study, we test the extent to which having more versus fewer siblings in the home environment may lead to differences in vocabulary development and the early linguistic environment over the course of the first 18 months of life.

In analyzing infants' lexical development in relation to the presence of older siblings in their household, the present work expands on the extant literature in two key ways. First, prior work generally considered birth order as a binomial factor (i.e.~comparing first-born infants with second-borns, e.g. Oshima-Takane \& Robbins, 2003), or `later-borns' (e.g. Hoff-Ginsberg, 1998), potentially missing graded effects. Instead of this approach, we consider the number of siblings, i.e.~how having \emph{more versus fewer siblings} is linked to an infant's lexical development and their early linguistic environment. Second, much of the existing literature in this area is drawn from questionnaire data or brief interactions recorded in the lab (but see Dunn \& Shatz, 1989 for a study of naturalistic home-recorded data), rather than naturalistic day-to-day interactions in the home. In contrast, we analyze an existing corpus of daylong audio- and hour-long video-recordings in concert with vocabulary checklists. Based on work summarized above, we expect that both the language environment and infants' early vocabulary will vary as a function of how many older siblings they have.

\hypertarget{hypotheses}{%
\section{Hypotheses}\label{hypotheses}}

Research has already shown that early lexical development is more advanced among first-born infants (e.g. Hoff-Ginsberg, 1998). We expect to see the same effect in our data, but we hypothesize that the closer granularity of this analysis will show a gradient decline in infants' lexical abilities in relation to an increasing number of siblings.
With regard to the infants' linguistic environment, we hypothesize that increased sibling number will have a negative effect on factors of the early input that are known to support language development. To test this, we adopt two input measures, established in the literature as being important for early language learning. The measures and our specific predictions are as follows:

\begin{enumerate}
\def\labelenumi{\arabic{enumi})}
\tightlist
\item
  \textbf{Amount of input} will be lower for children with more siblings. Following previous studies that show infants with siblings to receive less speech directed at them (Jones \& Adamson, 1987; Oshima-Takane \& Robbins, 2003), we expect to see the same effect in our sample. Given the link between amount of one-to-one input from the caregiver and vocabulary size (Ramírez-Esparza et al., 2014), we expect that infants who hear less input overall (i.e.~by our predictions, those with more siblings) will have smaller productive vocabularies.
\item
  \textbf{Amount of object presence} (word + object co-occurrence, e.g.~mother says `cat' when there is a cat in the room) will decrease as sibling number increases. As caregivers' attention is drawn away from one-to-one interactions with the infant, we expect there to be less opportunity for contingent talk and joint attention. The co-occurrence of words alongside their referents is thought to contribute to the earlier learning of nouns over verbs (Bergelson \& Swingley, 2013), as it supports the word learning process through the concrete mapping of word to referent (Gleitman, Cassidy, Nappa, Papafragou, \& Trueswell, 2005). We thus expect that increased sibling number will be associated with less object presence in caregiver speech, and subsequently a smaller productive vocabulary.
\end{enumerate}

\hypertarget{methods}{%
\section{Methods}\label{methods}}

We analyze data from the SEEDLingS corpus (Bergelson, Amatuni, Dailey, Koorathota, \& Tor, 2019), a longitudinal set of data incorporating home recordings, parental reports and experimental studies from the ages of 0;6 to 1;6. The present study draws on the parental report data to index child vocabulary size, and annotations of hour-long home video recordings, taken on a monthly basis during data collection, to index input. We also ran our input analysis using day-long audio recordings taken on a different day from the video data reported below; all results were consistent with those outlined below (see Supplementary Materials).

\hypertarget{participants}{%
\subsection{Participants}\label{participants}}

Forty-four families in New York State completed the year-long study. Infants (21 females) were from largely middle-class households; 33 mothers had attained a B.A. degree or higher. All infants had normal birthweight with no reported speech- or hearing-relevant diagnoses. Forty-two infants were Caucasian; two were from multi-racial backgrounds.

\hypertarget{materials}{%
\subsection{Materials}\label{materials}}

\textbf{Parental report data} To index each child's language abilities, we draw on data from vocabulary checklists {[}Macarthur-Bates Communicative Development Inventory, hereafter CDI; Fenson et al. (1994){]}, administered monthly from 0;6 to 1;6, along with a demographics questionnaire. Because the majority of infants did not produce their first word until around 0;11 according to CDI reports (M=10.70, SD=2.22)\footnote{Note that reported word production began earlier than observed word production (i.e.~in the video recordings) but this difference was not statistically significant (see Moore, Dailey, Garrison, Amatuni, \& Bergelson, 2019).}, we use CDI data from 0;10 on-wards in our analysis. CDI production data for each month is taken as a measure of the infants' lexical development over the course of the analysis period. Note that noun production in the home-recorded data correlates significantly with reported noun production in the CDI checklists from the age of 12 months, as reported in a previous study by Moore and colleagues (2019).

\textbf{Home-recorded video and audio data} Every month between 0;6 and 1;5, infants were video-recorded for one hour in their home, capturing a naturalistic representation of each infant's day-to-day input. Infants wore a hat with two small Looxcie video cameras attached, one pointed slightly up, and one pointed slightly down; this allowed us to record the scene from the infants' perspective. In the event that infants refused to wear the hats, caregivers wore the same kind of camera on a headband. Additionally, a camcorder was set up in the home. On a different day in the same month, the infants were recorded for upto 16 hours using a LENA recorder ({``{LENA} {Research} {Foundation},''} 2018) hidden in a small vest worn by the infant. Object words (i.e.~concrete nouns) deemed to be directed to or attended by the child were annotated by trained coders. Here we examine annotations for speaker, i.e.~who produced a word, and object presence, i.e.~whether the word's referent was present and attended to by the infant.

\hypertarget{procedure}{%
\subsection{Procedure}\label{procedure}}

We analyzed number of siblings based on parental report in the demographics questionnaires completed at 0;6 (R: 0-4). Siblings were on average 4.05 years older than the infants in this study (Mdn days: 1477, SD: 1477, R: 0-17 years).\footnote{For six infants, siblings' exact birthdates were not provided, and so age difference was estimated by subtracting the infant's age (6 months) from the sibling's age in years, as listed on the questionnaire (e.g.~if a sibling was 5 years old, they were classed as being 4.5 years older than the infant).} \emph{All siblings lived in the household with the infant; one infant had XX older siblings who lived in the household XX\% of the time. One family had a foster child live in the home for XX months of data collection.} All siblings were older than or of the same age as the infant in question.\footnote{Two infants in the dataset were dizygotic twins; our pattern of results holds when one of these infants is removed from the dataset.}

\hypertarget{input-measures}{%
\subsection{Input measures}\label{input-measures}}

Two input measures were derived based on the annotations of concrete nouns in this corpus, each pertaining to an aspect of the input that is established as important in early language learning: \textbf{overall household input} (how many concrete nouns does each infant hear?) and \textbf{object presence} (how much of this input is referentially transparent?). Each is described in further detail below.

\emph{Household Input} reflects how many nouns infants heard in the recordings from their mother, father and (where relevant) siblings. Other speakers' input was relatively rare during video recordings, accounting for 11.42\% of input overall (SD=22.82\%), and is excluded from our analysis. This measure of the early language environment is based on evidence showing strong links between the amount of speech heard in the early input and later vocabulary size (Anderson, Graham, Prime, Jenkins, \& Madigan, 2021). This analysis considers only nouns produced by speakers in the child's environment; concrete nouns are acquired earlier in development in English and cross-linguistically (Braginsky, Yurovsky, Marchman, \& Frank, 2019), and for this corpus of data, noun production correlates strongly with automated adult word count estimates (Bulgarelli \& Bergelson, 2020). Higher noun count in this data thus indicates higher input across the board.

\emph{Object Presence} was coded for each object word in the home recordings based on whether or not the annotator determined the object in question as present and attended to by the child. This is a metric of referential transparency, which has been suggested to aid in learning (Bergelson \& Swingley, 2013). Bergelson and Aslin (2017) found that word-object co-presence in the home correlated with infants' ability to recognise the same words in an eye-tracking experiment, suggesting an advantage for object labeling in word learning. This is consistent with findings from Cartmill and colleagues (2013), who found that more referentially-transparent interactions with the caregiver (as judged by adult speakers observing videos where target words were blanked out) predicted larger vocabulary size at 54 months. Indeed, presence of the labelled object decreases the ambiguity of the learning environment (Yurovsky, Smith, \& Yu, 2013), and may be a crucial component of supportive contingent talk (McGillion, Pine, Herbert, \& Matthews, 2017).

In the following analyses, we consider infants' productive vocabulary alongside our two input measures -- amount of household input and extent of object presence in the input -- as a function of sibling number. These measures index both input quality and quantity (though we make no distinction between quality/quantity of input here), and will be analysed in relation to infants' productive vocabulary (all word types included) as our dependent measure. Since the raw data are highly skewed, log-transformed data\footnote{1 was added to the raw infant production data of all infants before log-transformation to retain infants with vocabularies of 0.} and/or proportions are used for statistical analysis. All figures display non-transformed data for interpretive ease.

\hypertarget{results}{%
\section{Results}\label{results}}

Vocabulary development was highly variable across the 44 infants. By 18 months, 2 infants produced no words, while mean productive vocabulary size was 60.28 words (SD=78.31, Mdn=30.50). One female infant had a substantially larger reported vocabulary (3SDs above the mean monthly vocabulary score) between 1;1 and 1;6 and was classed as an outlier. We removed her from our data, leaving 43 infants (20 females) in the present analysis. Infants had one sibling on average (M=0.86, Mdn=1, SD=1.09). See Table \ref{tab:table-sibling-number}.

\begin{table}[tbp]

\begin{center}
\begin{threeparttable}

\caption{\label{tab:table-sibling-number}Sibling number by female and male infants.}

\small{

\begin{tabular}{llll}
\toprule
n Siblings & \multicolumn{1}{c}{Female} & \multicolumn{1}{c}{Male} & \multicolumn{1}{c}{Total}\\
\midrule
0 & 9 & 12 & 21\\
1 & 7 & 6 & 13\\
2 & 2 & 3 & 5\\
3 & 2 & 0 & 2\\
4 & 0 & 2 & 2\\
Total & 20 & 23 & 43\\
\bottomrule
\end{tabular}

}

\end{threeparttable}
\end{center}

\end{table}

\hypertarget{model-structure-for-fixed-and-random-effects}{%
\subsection{Model structure for fixed and random effects}\label{model-structure-for-fixed-and-random-effects}}

All reported models were generated in R (R Core Team, 2019) using the \emph{lmerTest} package to run linear mixed-effects regression models (Kuznetsova, Brockhoff, \& Christensen, 2017). P-values were generated by likelihood ratio tests resulting from nested model comparison. All models include infant as a random effect. All post-hoc tests are two-sample, two-tailed Wilcoxon Tests; given that all of our variables differed significantly from normal by Shapiro tests, we opted to run non-parametric tests for all post-hoc comparisons. Where multiple post-hoc comparisons are run on the same dataset, Bonferroni corrections are applied with an adjusted threshold of \(a\)=0.03, accounting for two between-group comparisons (no siblings vs.~one sibling, one sibling vs.~multiple siblings; see below). Our dataset included 12 hours of home-recorded video data and 12 vocabulary questionnaires per child (as well as 12 days of audio recordings). While this data provides a substantial representation of each child's early language environment and development, we acknowledge that it is nevertheless limited in terms of group size, and means we cannot effectively combine analyses of sibling number alongside other measures such as sex.

Before considering sibling status, we first modeled infants' productive vocabulary at 18 months (taken from CDI questionnaires) as a function of age, sex, and mother's education. There was no effect of sex (\emph{p}=.632)\footnote{Though note that Dailey and Bergelson (under review) find a sex difference in the number of word types produced in the home-recorded data between 6 and 17 months.}, and no correlation with mothers' education level (across five categories from High School to Doctorate; \emph{r} = -0.01, \emph{p}=.139). As expected, age had a significant effect on productive vocabulary (\emph{p}\textless{} .001), and so we include age as a fixed effect in our models. Because we expected that maternal age and education might have an effect on both sibling number and infant productive vocabulary, we ran further correlations to test these variables. There was no correlation between mother's education and number of siblings (\emph{r} = -0.01, \emph{p}=.928), and a marginal positive correlation between mother's age and number of siblings (Spearman's \emph{r} = 0.28, \emph{p}=.069); older mothers tended to have more children. However, no correlation was found between mothers' age and productive vocabulary at 18 months (\emph{r} = -0.04, \emph{p}=.822).

\hypertarget{effect-of-siblings-on-infants-productive-vocabulary}{%
\subsection{Effect of siblings on infants' productive vocabulary}\label{effect-of-siblings-on-infants-productive-vocabulary}}

We next modeled the effect of siblings on reported productive vocabulary. We tested three variables representing the sibling effect: a binary variable (0 vs.~\textgreater0 siblings), aggregated groups (None vs.~One vs.~2+), and discrete sibling number (0 vs.~1 vs.~2 vs.~3 vs.~4 siblings) using the following nested model structure, where (1) is the baseline model, and (2) includes siblings as the variable of interest:

\begin{enumerate}
\def\labelenumi{\arabic{enumi}.}
\tightlist
\item
  vocabulary size (log-transformed) \textasciitilde{} age (months) + (1\textbar subj)
\item
  vocabulary size (log-transformed) \textasciitilde{} siblings {[}binary, group or discrete{]} + age (months) + (1\textbar subj)
\end{enumerate}

In our sample, the simple fact of having siblings (i.e.~as a binary variable) did not affect reported CDI vocabulary size, while both discrete sibling number and sibling group did. See Table \ref{tab:table-sibling-model-output}.

\begin{table}[H]

\begin{center}
\begin{threeparttable}

\caption{\label{tab:table-sibling-model-output}Output from likelihood ratio tests comparing regression models that predict the effects of sibling number (binary, grouped and discrete variables) on vocabulary size. Month was included in each model as a fixed effect; subject was included as a random effect.}

\small{

\begin{tabular}{llll}
\toprule
Model & \multicolumn{1}{c}{Df} & \multicolumn{1}{c}{Chisq} & \multicolumn{1}{c}{p value}\\
\midrule
0 vs. >0 siblings & 1.00 & 2.27 & 0.13\\
Sibling group & 2.00 & 7.96 & 0.02\\
Sibling number & 1.00 & 6.24 & 0.01\\
\bottomrule
\end{tabular}

}

\end{threeparttable}
\end{center}

\end{table}

\begin{longtable}[t]{cccccc}
\caption{\label{tab:table-sibgroup-model-summary}Full model output from linear mixed effects regression model comparing language development over time in relation to sibling group. Age in months was included as a fixed effect; subject was included as a random effect.}\\
\toprule
Effect & Estimate & Std. Error & df & t value & p\\
\midrule
Intercept & 0.70 & 0.20 & 68.30 & 3.43 & 0.001\\
SibGroupOne & -0.05 & 0.29 & 43.10 & -0.17 & 0.863\\
SibGroup2+ & -0.94 & 0.33 & 43.88 & -2.83 & 0.007\\
month11 & 0.38 & 0.14 & 321.34 & 2.69 & 0.007\\
month12 & 0.77 & 0.14 & 321.64 & 5.39 & <0.001\\
\addlinespace
month13 & 1.07 & 0.14 & 321.58 & 7.60 & <0.001\\
month14 & 1.39 & 0.14 & 321.45 & 9.93 & <0.001\\
month15 & 1.69 & 0.14 & 321.54 & 11.96 & <0.001\\
month16 & 2.03 & 0.14 & 321.54 & 14.40 & <0.001\\
month17 & 2.45 & 0.14 & 321.79 & 16.96 & <0.001\\
\addlinespace
month18 & 2.82 & 0.15 & 322.13 & 19.07 & <0.001\\
\bottomrule
\end{longtable}

Having more siblings was associated with a smaller vocabulary size over the course of early development. This is consistent with previous findings (Hoff-Ginsberg, 1998; Pine, 1995). Moreover, for each additional sibling, infants were reported to have acquired 30.59\% fewer words. The `sibling effect' is thus present in our data. The grouped sibling variable (0 vs.~1 vs.~2+) was selected as our measure of siblings as it allows analysis across more equal group sizes. However, note that all regression reported below were consistent when discrete sibling number was analysed (see Supplementary Data).

According to CDI reports, infants with one sibling acquire only 5.07\% fewer words than firstborns over the course of our analysis, while infants with two or more siblings produce 93.73\% fewer words. See Tables \ref{tab:table-sibgroup-model-summary} and \ref{tab:table-data-summary}, and Figure \ref{fig:Figure-SibGroup}. Post-hoc Wilcoxon Rank Sum tests comparing reported productive vocabulary at 18 months revealed significantly larger vocabularies for infants with one sibling compared to those with two or more siblings (\emph{W}=5, \emph{p} = .004), but no difference between infants with one sibling and those with no siblings (\emph{W}=79.50, \emph{p} = .631).

\begin{figure}
\centering
\includegraphics{SiblingsStudyText_files/figure-latex/Figure-SibGroup-1.pdf}
\caption{\label{fig:Figure-SibGroup}Reported productive vocabulary acquisition (CDI) over time. Colors denote sibling group; line with colored confidence band reflects local estimator (loess) fit over individual infants' vocabulary at each month. Triangles indicate mean with bootstrapped CIs computed over each month's data. Points (jittered horizontally) show individual infants' vocabulary size at each month. Y-axis utilizes log-transformed vertical spacing for visual clarity.}
\end{figure}

\hypertarget{effect-of-siblings-on-infants-input}{%
\subsection{Effect of siblings on infants' input}\label{effect-of-siblings-on-infants-input}}

Having established that infants' productive vocabulary varied as a function of how many siblings they had, we turn to our input measures to test whether input varied by a child's sibling status. To keep relatively similar Ns across groups we used the 0 vs.~1 vs.~2+ siblings division. That said, with the exception of household input (see Supplementary Data) all reported model outcomes hold if we model discrete sibling number as a fixed effect instead.

As with our previous analysis, we first modeled infants' input (maternal input only) as a function of age, sex and maternal education. This time, there was no effect for age, nor sex or maternal education (\emph{p}s all\textgreater.260) on the amount of input produced by mothers. We therefore excluded all three variables from our models.

\hypertarget{parental-input}{%
\subsubsection{Parental Input}\label{parental-input}}

\begin{table}

\caption{\label{tab:table-data-summary}Data summary of all three input variables and reported vocabulary size at 18 months.}
\centering
\begin{tabular}[t]{ccccccc}
\toprule
\multicolumn{1}{c}{ } & \multicolumn{2}{c}{No siblings} & \multicolumn{2}{c}{1 sibling} & \multicolumn{2}{c}{2+ siblings} \\
\cmidrule(l{3pt}r{3pt}){2-3} \cmidrule(l{3pt}r{3pt}){4-5} \cmidrule(l{3pt}r{3pt}){6-7}
\multicolumn{1}{c}{Variable} & \multicolumn{1}{c}{Mean} & \multicolumn{1}{c}{SD} & \multicolumn{1}{c}{Mean} & \multicolumn{1}{c}{SD} & \multicolumn{1}{c}{Mean} & \multicolumn{1}{c}{SD} \\
\cmidrule(l{3pt}r{3pt}){1-1} \cmidrule(l{3pt}r{3pt}){2-2} \cmidrule(l{3pt}r{3pt}){3-3} \cmidrule(l{3pt}r{3pt}){4-4} \cmidrule(l{3pt}r{3pt}){5-5} \cmidrule(l{3pt}r{3pt}){6-6} \cmidrule(l{3pt}r{3pt}){7-7}
\% object presence in input & 0.67 & 0.15 & 0.56 & 0.15 & 0.46 & 0.18\\
N Input utterances, 10-17 months & 180.63 & 124.85 & 184.43 & 84.76 & 100.19 & 52.80\\
Productive Vocabulary 18m (CDI) & 58.89 & 60.76 & 92.64 & 111.42 & 13.00 & 9.49\\
\bottomrule
\end{tabular}
\end{table}

\begin{figure}
\centering
\includegraphics{SiblingsStudyText_files/figure-latex/Figure-Speaker-count-1.pdf}
\caption{\label{fig:Figure-Speaker-count}Mean number of words produced by Mothers, Fathers and Siblings, as well as total family input (mother + father + sibling(s)), across sessions recorded between 10-17 months. Circles represent values for individual infants; red triangles show group means.}
\end{figure}

Mothers provided the largest proportion of the infants' overall input across the sample (80\%, M=146.10 object words, Mdn=125, SD=119.97). Fathers accounted for an average of 14\% (M=22.13, Mdn=0, SD=48.31), while infants with siblings received around 6\% of their input from their brothers and sisters (M=16.18, Mdn=11, SD=18.51). See Table \ref{tab:table-data-summary} and Figure \ref{fig:Figure-Speaker-count}. We tested overall quantity of input (aggregated across mothers, fathers, and siblings) in our model, and a significant effect was found (\(\chi^2 (2)\) = 18.48, \emph{p}\textless{} .001). We then ran post-hoc tests to compare mean amount of input across sibling groups; these showed a significant difference in average input received between infants with one sibling versus those with two
or more siblings (\emph{W}=7, \emph{p}\textless{} .001, \(a\)=0.03) while amount of input did not differ between infants with no siblings and those with one sibling (\emph{W}=120, \emph{p} = .576). On average, in any given hour-long recording, infants with one siblings heard 4 \textbf{more} object words in their input than those with no sibling, and 89 more than those with two or more siblings. Infants with one sibling heard 94 more object words than those with two or more siblings.

Next, we tested how much of that input came from siblings (for infants who had them). Overall, for infants who had siblings, at least one other child was present in 72.16\% of recordings (n = 176). Wilcoxon Rank Sum tests showed no difference between the amount of sibling input received by infants with one sibling compared with those with two or more siblings (\emph{W}=40, \emph{p}=.235), contrasting with predictions set out in our first hypothesis. Looking at caregivers individually, infants with two or more siblings heard significantly less input from their mothers than those with one sibling (\emph{W}=15, \emph{p}=.003), while there was no difference between those with one vs.~no siblings (\emph{W}=126, \emph{p} = .727). Finally, amount of paternal input did not differ between groups (one vs.~none: \emph{W}=152, \emph{p} = .606; one vs.~2+: \emph{W}=42, \emph{p}=.296).

\hypertarget{object-presence}{%
\subsubsection{Object presence}\label{object-presence}}

\begin{figure}
\centering
\includegraphics{SiblingsStudyText_files/figure-latex/Figure-object-presence-1.pdf}
\caption{\label{fig:Figure-object-presence}Proportion of input words produced with object presence across sibling groups. Error bars and black triangles show 95\% CIs and mean proportion of object presence across sibling groups. Dots indicate mean proportion of object presence per infant, collapsing across age and jittered horizontally for visual clarity.}
\end{figure}

On average, 60\% of utterances were produced in the presence of the relevant object (Mdn=0.60, SD=0.12). We hypothesized that infants with more siblings would hear fewer words in referentially transparent conditions (i.e.~they would experience lower object presence) than those with fewer siblings. Indeed, modelling the quantity of object present tokens that infants heard, we find a significant effect for sibling group on object presence (\(\chi^2 (2)\) = 26.09, \emph{p}\textless{} .001). See Figure \ref{fig:Figure-object-presence}. Infants with no siblings experienced 23\% more object presence in their input than those with two or more siblings, and 12\% more than those with one sibling. Post-hoc comparisons revealed significant between-group differences: infants with no siblings experienced significantly more object presence than those with one sibling (\emph{W}=234, \emph{p}\textless{} .001, \(a\)=0.03). Likewise, infants with one sibling experienced significantly more object presence those with two or more siblings (\emph{W}=20, \emph{p}=.009). See Table \ref{tab:table-data-summary}.

\hypertarget{discussion}{%
\section{Discussion}\label{discussion}}

We investigated the nature of infant language development in relation to number of children in the household. Previous research found a delay in lexical acquisition for later-born infants (Fenson et al., 1994; Hoff, 2006), with differences in input across birth order reported as a root cause. Our results add several new dimensions to this, by testing for differences across more vs.~fewer older siblings, and by looking at input during child-centered home recordings. Infants with more siblings were reported to say fewer words by 18 months, heard fewer nouns from their parents and siblings, and experienced less `object co-presence' when hearing them.

Importantly, and in contrast with some previous research (Hoff-Ginsberg, 1998; Oshima-Takane \& Robbins, 2003), infants with one sibling showed no delay in lexical production and minimal reduction in input in comparison to first-born infants. That is, our results suggest that simply having a sibling does not contribute to input or vocabulary differences across children (as measured here), while having more than one siblings seems to do so. Indeed, infants with zero and one sibling had similar results for productive vocabulary, and parental noun input overall, but not object presence, which we return to below. In contrast, infants with two or more siblings said fewer words, and also heard fewer input words with proportionally less object co-presence, compared with their peers.

When we considered the effect of sibling status -- that is, whether or not infants had any siblings, disregarding specific sibling number -- our findings showed that having siblings made no difference to infants' lexical production capacities. This contrasts with Hoff-Ginsberg (1998), who found that, by 18 months, laterborns exhibit lower language skills. However, Oshima-Takane and colleagues (1996) found no overall differences between first- and second-born children across a range of language measures taken at 21 months. Our finer-grained results suggest a greater role for \emph{sibling quantity} over first- vs.~later-born status. The more older siblings a child had, the lower their reported productive vocabulary at 18 months. This adds to findings from Fenson and colleagues (1994), who found a weak but significant negative correlation between birth order and production of both words and gestures. Controlling for age, our model showed that for each additional older sibling, infants produced more than 30\% fewer words by 18 months.
While infants with more siblings heard less input speech overall, having one sibling did not significantly reduce the number of nouns in an infant's input. This is in direct contrast with reports from the literature; Hoff (2006) states that ``when a sibling is present, each child receives less speech directed solely at\ldots her \emph{because mothers produce the same amount of speech whether interacting with one or two children}'' (p.67, italics added). While this does not appear to be the case in the present dataset, it may be due to the circumstances of the home-recorded data: while siblings were present in many of the recordings (72.16\% of recordings in which the target child had a sibling), given the focus of the data collection, parents may have had a tendency to direct their attention - and consequently their linguistic input - more towards the target child. Alternatively, our results may diverge from those of Hoff (2006) due to the nature of our input measure, which only took nouns into account. However, Bulgarelli and Bergelson (2020) show that nouns are a reliable proxy for overall input in this dataset, thus suggesting that this measure provides an appropriate representation of overall input directed at the target child.

Moreover, infants with siblings did not hear much speech from their older brothers and sisters. This is contrary to our hypothesis, as having more siblings did not predict more sibling input. Similar findings are reported in a lab-based interaction study by Oshima-Takane and Robbins (2003), who found that older siblings rarely talked directly to the target child; instead, most input from siblings was overheard speech from sibling-mother interactions. However, results from Havron and colleagues (2019) indirectly suggest that speech from siblings may affect language development, and not necessarily in a negative direction. They found that children with older brothers had lower verbal skills than children with no siblings; children with older sisters did not show this effect. The authors propose that this differential effect could be due sisters having positive effects on language development (i.e., the effect is derived from supportive sibling input), or perhaps due to brothers' additional demands on caregiver time and attention, thus directing caregiver attention away from the target child (i.e., the effect is derived from changes to caregiver input). We did not analyse sibling sex in our data, but future analyses could consider input speech in relation to sibling sex.

The `sibling effect' was most marked in our analysis of object presence. In this case, even having one sibling led to fewer word-object pairs presented in the input. Presence of a labeled object with congruent input speech is known to be supportive in early word learning: Bergelson and Aslin (2017) combined analysis of this home-recorded data with an experimental study to show that word-object co-presence in naturalistic caregiver input supported comprehension of nouns when tested using eye-tracking. Gogate and colleagues (2000) state that contingent word production supports the learning of novel word-object combinations, as ``multimodal motherese'' - whereby a target word is produced in synchrony with its referent, often involving movement or touch of the object - supports word learning by demonstrating novel word-object combinations. Indeed, lower rates of referential transparency in children's input have also been proposed to explain why common non-nouns like \emph{hi} and \emph{uh-oh} are learned later than concrete nouns (Bergelson \& Swingley, 2013).

Object presence varied more linearly across sibling quantity, suggesting it may be a less critical driver of early word production. Given that infants with one sibling heard approximately the same number of object words in the input than those with no siblings, input, rather than object presence, may be a more crucial factor in predicting a child's vocabulary size by 18 months. Alternatively, the reduced object presence for children with one sibling may have been compensated for in other ways we did not measure here, which in turn resulted in the indistinguishable vocabulary difference in the 0 and 1 sibling children at 18 months.

More generally, one possibility raised by these results is that perhaps parents are able to compensate or provide relatively similar input and learning support for one or two children, but once children outnumber parents, this balancing act of attention, care, and time, becomes unwieldy. While the current sample is relatively limited and homogenous in the family structures and demographics it includes, future work could fruitfully investigate this possibility by considering whether (controlling for other potential contributors like SES, Hoff-Ginsberg, 1998) the presence of more caregivers (whether parents, relatives, or other adults) helps foster language development.

Alternatively, second-borns might `even out' with children with no siblings due to a trade-off between direct attention from the caregiver and the possibility of more sophisticated social-communicative interactions. For these infants there is still ample opportunity to engage with the mother in one-to-one interactions, allowing a higher share of her attention than is available to third- or later-borns. Furthermore, triadic interactions can benefit the development of a number of linguistic and communication skills (Barton \& Tomasello, 1991; Dunn \& Shatz, 1989). Second-borns may also benefit from overheard speech in their input, supporting the acquisition of nouns and even more complex lexical categories (Floor \& Akhtar, 2006; Oshima-Takane et al., 1996). For infants with one sibling, the benefits of observing/overhearing interactions between sibling and caregiver, as well as the possibility for partaking in such interactions, may outweigh the decrease in some aspects of the input (i.e., in our data, only observed in object presence). Having more than one sibling may throw this off-balance.

Importantly, the present results make no claims about eventual outcomes for these children: generally speaking, regardless of sibling number, all typically-developing infants reach full and fluent language use. Indeed, some research suggests that sibling effects, while they may be clear in early development, are not always sustained into childhood; e.g.~twins are known to experience a delay in language development into the third year, but are quick to catch up thereafter (Dales, 1969; Tomasello, Mannle, \& Kruger, 1986). This demonstrates the cognitive adaptability of early development, which brings about the acquisition of language across varying and allegedly `imperfect' learning environments. Infants' capacity to develop linguistic skills from the resources that are available to them -- whether that is infant-directed object labels or overheard abstract concepts -- highlights the dynamic and adaptable nature of early cognitive development, and a system that is sufficiently robust to bring about the same outcome across populations.

Of course, the `success' of early language development is defined by how success is measured. Here we chose word production as our measure of linguistic capability; we did not consider other equally valid measures such as language comprehension or early social-interaction skills. Similarly, our input measures focused on nouns; other lexical classes may reveal different effects, though they are generally sparser until toddlerhood. There is also some imbalance in group sizes across our data; our sample was not pre-selected for sibling number, and so group sizes are unmatched across the analysis. Including a larger number of infants with 2+ siblings may have revealed a different pattern of results. Finally, more work across wider and larger populations is necessary to unpack the generalizability of the present results. Our sample is refelective of average household sizes in middle-class families across North America and Western Europe (Office for National Statistics, 2018; United States Census Bureau, 2010), but it is not unusual in some communities and parts of the world for households to include between three and six children on average (Institute for Family Studies \& Wheatley Institution, 2019). Adding to this, it is also necessary to consider cross-cultural differences in the way children are addressed by their parents. Casillas, Brown and Levinson (2019) found that almost all of Tseltal Mayan children's input came from speech directed at other people (21 minutes per hour, compared with just under 4 minutes/hour of specifically child-directed input), while Shneidman and Goldin-Meadow (2012) found that 69\% of speech directed at Mayan children came from siblings (in comparison with 10\% for children in the USA).

In conclusion, our results support the general findings from the literature showing that later-born infants have slower lexical acquisition than their first-born peers. However, we highlight an important difference from previous findings, namely that in the present sample, second-born infants show no such effect, while infants with more than two siblings have significantly smaller vocabularies at age 18 months. We related this directly to the infants' input over a period of one year. Future studies should consider the granularity of more versus fewer siblings, and how this relates to language abilities over the course of development.

\newpage

\hypertarget{references}{%
\section{References}\label{references}}

\begingroup
\setlength{\parindent}{-0.5in}
\setlength{\leftskip}{0.5in}

\hypertarget{refs}{}
\begin{CSLReferences}{1}{0}
\leavevmode\vadjust pre{\hypertarget{ref-anderson_linking_2021}{}}%
Anderson, N. J., Graham, S. A., Prime, H., Jenkins, J. M., \& Madigan, S. (2021). Linking {Quality} and {Quantity} of {Parental} {Linguistic} {Input} to {Child} {Language} {Skills}: {A} {Meta}-{Analysis}. \emph{Child Development}, \emph{92}(2), 484--501. \url{https://doi.org/10.1111/cdev.13508}

\leavevmode\vadjust pre{\hypertarget{ref-barton_joint_1991}{}}%
Barton, M. E., \& Tomasello, M. (1991). Joint {Attention} and {Conversation} in {Mother}-{Infant}-{Sibling} {Triads}, \emph{62}(3), 517--529.

\leavevmode\vadjust pre{\hypertarget{ref-bergelson_day_2019}{}}%
Bergelson, E., Amatuni, A., Dailey, S., Koorathota, S., \& Tor, S. (2019). Day by day, hour by hour: {Naturalistic} language input to infants. \emph{Developmental Science}, \emph{22}(1), e12715. \url{https://doi.org/10.1111/desc.12715}

\leavevmode\vadjust pre{\hypertarget{ref-bergelson_nature_2017}{}}%
Bergelson, E., \& Aslin, R. N. (2017). Nature and origins of the lexicon in 6-mo-olds. \emph{Proceedings of the National Academy of Sciences}, \emph{114}(49), 12916--12921. \url{https://doi.org/10.1073/pnas.1712966114}

\leavevmode\vadjust pre{\hypertarget{ref-bergelson_acquisition_2013}{}}%
Bergelson, E., \& Swingley, D. (2013). The {Acquisition} of {Abstract} {Words} by {Young} {Infants}. \emph{Cognition}, \emph{127}(3), 391--397. \url{https://doi.org/10.1038/jid.2014.371}

\leavevmode\vadjust pre{\hypertarget{ref-berglund_communicative_2005}{}}%
Berglund, E., Eriksson, M., \& Westerlund, M. (2005). Communicative skills in relation to gender, birth order, childcare and socioeconomic status in 18-month-old children. \emph{Scandinavian Journal of Psychology}, \emph{46}(6), 485--491. \url{https://doi.org/10.1111/j.1467-9450.2005.00480.x}

\leavevmode\vadjust pre{\hypertarget{ref-braginsky_consistency_2019}{}}%
Braginsky, M., Yurovsky, D., Marchman, V. A., \& Frank, M. C. (2019). Consistency and {Variability} in {Children}'s {Word} {Learning} {Across} {Languages}. \emph{Open Mind}, \emph{3}, 52--67. \url{https://doi.org/10.1162/opmi_a_00026}

\leavevmode\vadjust pre{\hypertarget{ref-bulgarelli_look_2020}{}}%
Bulgarelli, F., \& Bergelson, E. (2020). Look who's talking: {A} comparison of automated and human-generated speaker tags in naturalistic day-long recordings. \emph{Behavior Research Methods}, \emph{52}(2), 641--653. \url{https://doi.org/10.3758/s13428-019-01265-7}

\leavevmode\vadjust pre{\hypertarget{ref-cartmill_quality_2013}{}}%
Cartmill, E. a., Armstrong, B. F., Gleitman, L. R., Goldin-Meadow, S., Medina, T. N., \& Trueswell, J. C. (2013). Quality of early parent input predicts child vocabulary 3 years later. \emph{Proceedings of the National Academy of Sciences of the United States of America}. \url{https://doi.org/10.1073/pnas.1309518110}

\leavevmode\vadjust pre{\hypertarget{ref-casillas_early_2019}{}}%
Casillas, M., Brown, P., \& Levinson, S. C. (2019). Early {Language} {Experience} in a {Tseltal} {Mayan} {Village}. \emph{Child Development}, \emph{EarlyView article}. \url{https://doi.org/10.1111/cdev.13349}

\leavevmode\vadjust pre{\hypertarget{ref-dailey_talking_nodate}{}}%
Dailey, S., \& Bergelson, E. (under review). Talking to talkers: {Infants}' talk status, but not their gender, is related to language input.

\leavevmode\vadjust pre{\hypertarget{ref-dales_motor_1969}{}}%
Dales, R. J. (1969). Motor and language development of twins during the first three years. \emph{The Journal of Genetic Psychology; Provincetown, Mass., Etc.}, \emph{114}(2), 263--271. Retrieved from \url{https://search.proquest.com/docview/1297124434/citation/D928716F9A7E4AEFPQ/1}

\leavevmode\vadjust pre{\hypertarget{ref-dunn_becoming_1989}{}}%
Dunn, J., \& Shatz, M. (1989). Becoming a {Conversationalist} despite ({Or} {Because} of) {Having} an {Older} {Sibling}. \emph{Child Development}, \emph{60}(2), 399--410.

\leavevmode\vadjust pre{\hypertarget{ref-fenson_variability_1994}{}}%
Fenson, L., Dale, P. S., Reznick, J. S., Bates, E., Thal, D. J., Pethick, M., Stephen J. Tomasello, \ldots{} Stiles, J. (1994). Variability in {Early} {Communicative} {Development}. \emph{Monographs of the Society for Research in Child Development}, \emph{59}. \url{https://doi.org/10.2307/1166093}

\leavevmode\vadjust pre{\hypertarget{ref-floor_can_2006}{}}%
Floor, P., \& Akhtar, N. (2006). Can 18-{Month}-{Old} {Infants} {Learn} {Words} by {Listening} {In} on {Conversations}? \emph{Infancy}, \emph{9}(3), 327--339. \url{https://doi.org/10.1207/s15327078in0903_4}

\leavevmode\vadjust pre{\hypertarget{ref-gleitman_hard_2005}{}}%
Gleitman, L. R., Cassidy, K., Nappa, R., Papafragou, A., \& Trueswell, J. C. (2005). Hard {Words}. \emph{Language Learning and Development}, \emph{1}(1), 23--64. \url{https://doi.org/10.1207/s15473341lld0101_4}

\leavevmode\vadjust pre{\hypertarget{ref-gogate_study_2000}{}}%
Gogate, L. J., Bahrick, L. E., \& Watson, J. D. (2000). A {Study} of {Multimodal} {Motherese}: {The} {Role} of {Temporal} {Synchrony} between {Verbal} {Labels} and {Gestures}. \emph{Child Development}, \emph{71}(4), 878--894. \url{https://doi.org/10.1111/1467-8624.00197}

\leavevmode\vadjust pre{\hypertarget{ref-havron_effect_2019}{}}%
Havron, N., Ramus, F., Heude, B., Forhan, A., Cristia, A., Peyre, H., \ldots{} Thiebaugeorges, O. (2019). The {Effect} of {Older} {Siblings} on {Language} {Development} as a {Function} of {Age} {Difference} and {Sex}. \emph{Psychological Science}, \emph{30}(9), 1333--1343. \url{https://doi.org/10.1177/0956797619861436}

\leavevmode\vadjust pre{\hypertarget{ref-hoff_how_2006}{}}%
Hoff, E. (2006). How social contexts support and shape language development. \emph{Developmental Review}, \emph{26}(1), 55--88. \url{https://doi.org/10.1016/j.dr.2005.11.002}

\leavevmode\vadjust pre{\hypertarget{ref-hoff-ginsberg_relation_1998}{}}%
Hoff-Ginsberg, E. (1998). The relation of birth order and socioeconomic status to children's language experience and language development. \emph{Applied Psycholinguistics}, \emph{19}(4), 603--629. \url{https://doi.org/10.1017/S0142716400010389}

\leavevmode\vadjust pre{\hypertarget{ref-institute_for_family_studies_world_2019}{}}%
Institute for Family Studies, \& Wheatley Institution. (2019). \emph{World family map 2019: {Mapping} family change and child well-being outcomes}. Charlottesville, VA: Institute for Family Studies. Retrieved from \url{https://ifstudies.org/reports/world-family-map/2019/executive-summary}

\leavevmode\vadjust pre{\hypertarget{ref-jones_language_1987}{}}%
Jones, C. P., \& Adamson, L. B. (1987). Language {Use} in {Mother}-{Child} and {Mother}-{Child}-{Sibling} {Interactions}, \emph{58}(2), 356--366.

\leavevmode\vadjust pre{\hypertarget{ref-kuznetsova_lmertest_2017}{}}%
Kuznetsova, A., Brockhoff, P. B., \& Christensen, R. H. B. (2017). \{{lmerTest}\} {Package}: {Tests} in {Linear} {Mixed} {Effects} {Models}. \emph{Journal of Statistical Software}, \emph{82}(13), 1--26. \url{https://doi.org/10.18637/jss.v082.i13}

\leavevmode\vadjust pre{\hypertarget{ref-noauthor_lena_2018}{}}%
{LENA} {Research} {Foundation}. (2018). Retrieved from \url{https://www.lena.org/}

\leavevmode\vadjust pre{\hypertarget{ref-mcgillion_randomised_2017}{}}%
McGillion, M., Pine, J. M., Herbert, J. S., \& Matthews, D. (2017). A randomised controlled trial to test the effect of promoting caregiver contingent talk on language development in infants from diverse socioeconomic status backgrounds. \emph{Journal of Child Psychology and Psychiatry}, \emph{58}(10), 1122--1131. \url{https://doi.org/10.1111/jcpp.12725}

\leavevmode\vadjust pre{\hypertarget{ref-moore_point_2019}{}}%
Moore, C., Dailey, S., Garrison, H., Amatuni, A., \& Bergelson, E. (2019). Point, {Walk}, {Talk}: {Links} {Between} {Three} {Early} {Milestones}, {From} {Observation} and {Parental} {Report}. \emph{Developmental Psychology}. \url{https://doi.org/10.1037/dev0000738}

\leavevmode\vadjust pre{\hypertarget{ref-office_for_national_statistics_families_2018}{}}%
Office for National Statistics. (2018). \emph{Families with dependent children by number of children, {UK}, 1996 to 2017} (No. 008855). Office for National Statistics. Retrieved from \url{https://www.ons.gov.uk/peoplepopulationandcommunity/birthsdeathsandmarriages/families/adhocs/008855familieswithdependentchildrenbynumberofchildrenuk1996to2017}

\leavevmode\vadjust pre{\hypertarget{ref-oshima-takane_birth_1996}{}}%
Oshima-Takane, Y., Goodz, E., \& Derevensky, J. L. (1996). Birth {Order} {Effects} on {Early} {Language} {Development} : {Do} {Secondborn} {Children} {Learn} from {Overheard} {Speech} ? {Author} ( s ): {Yuriko} {Oshima}-{Takane} , {Elizabeth} {Goodz} and {Jeffrey} {L} . {Derevensky} {Published} by : {Wiley} on behalf of the {Society} for {Research} in {Child} {De}, \emph{67}(2), 621--634.

\leavevmode\vadjust pre{\hypertarget{ref-oshima-takane_linguistic_2003}{}}%
Oshima-Takane, Y., \& Robbins, M. (2003). Linguistic environment of secondborn children. \emph{First Language}, \emph{23}(1), 21--40. https://doi.org/\url{http://dx.doi.org/10.1177/0142723703023001002}

\leavevmode\vadjust pre{\hypertarget{ref-pine_variation_1995}{}}%
Pine, J. M. (1995). Variation in {Vocabulary} {Development} as a {Function} of {Birth} {Order}. \emph{Child Development}, \emph{66}(1), 272--281.

\leavevmode\vadjust pre{\hypertarget{ref-r_core_team_r_2019}{}}%
R Core Team. (2019). R: {A} {Language} {Environment} for {Statistical} {Computing}. R Foundation for Statistical Computing. Retrieved from \url{https://www.R-project.org/}

\leavevmode\vadjust pre{\hypertarget{ref-ramirez-esparza_look_2014}{}}%
Ramírez-Esparza, N., García-Sierra, A., \& Kuhl, P. K. (2014). Look who's talking: Speech style and social context in language input to infants are linked to concurrent and future speech development. \emph{Developmental Science}, \emph{17}(6), 880--891. \url{https://doi.org/10.1016/j.surg.2006.10.010.Use}

\leavevmode\vadjust pre{\hypertarget{ref-shneidman_language_2012}{}}%
Shneidman, L. A., \& Goldin‐Meadow, S. (2012). Language input and acquisition in a {Mayan} village: How important is directed speech? \emph{Developmental Science}, \emph{15}(5), 659--673. \url{https://doi.org/10.1111/j.1467-7687.2012.01168.x}

\leavevmode\vadjust pre{\hypertarget{ref-soderstrom_beyond_2007}{}}%
Soderstrom, M. (2007). Beyond babytalk: {Re}-evaluating the nature and content of speech input to preverbal infants. \emph{Developmental Review}, \emph{27}(4), 501--532. \url{https://doi.org/10.1016/j.dr.2007.06.002}

\leavevmode\vadjust pre{\hypertarget{ref-stern_prosody_1983}{}}%
Stern, D. N., Spieker, S., Barnett, R. K., \& MacKain, K. (1983). The {Prosody} of {Maternal} {Speech}: {Infant} {Age} and {Context} {Related} {Changes}. \emph{Journal of Child Language}, \emph{10}(1), 1--15. \url{https://doi.org/10.1017/S0305000900005092}

\leavevmode\vadjust pre{\hypertarget{ref-tomasello_linguistic_1986}{}}%
Tomasello, M., Mannle, S., \& Kruger, A. C. (1986). Linguistic environment of 1- to 2-year-old twins. \emph{Developmental Psychology}, \emph{22}(2), 169--176. \url{https://doi.org/10.1037/0012-1649.22.2.169}

\leavevmode\vadjust pre{\hypertarget{ref-united_states_census_bureau_household_2010}{}}%
United States Census Bureau. (2010). \emph{Household {Type} by {Number} of {People} {Under} 18 {Years}} (No. PCT16). Retrieved from \url{https://data.census.gov/cedsci/table?q=number\%20of\%20children\&hidePreview=false\&tid=DECENNIALSF12010.PCT16\&t=Children\&vintage=2018}

\leavevmode\vadjust pre{\hypertarget{ref-yurovsky_statistical_2013}{}}%
Yurovsky, D., Smith, L. B., \& Yu, C. (2013). Statistical word learning at scale: The baby's view is better. \emph{Developmental Science}, \emph{16}(6), 959--966. \url{https://doi.org/10.1111/desc.12036}

\leavevmode\vadjust pre{\hypertarget{ref-zambrana_impact_2012}{}}%
Zambrana, I. M., Ystrom, E., \& Pons, F. (2012). Impact of {Gender}, {Maternal} {Education}, and {Birth} {Order} on the {Development} of {Language} {Comprehension}: {A} {Longitudinal} {Study} from 18 to 36 {Months} of {Age}. \emph{Journal of Developmental \& Behavioral Pediatrics}, \emph{33}(2), 146--155. \url{https://doi.org/10.1097/DBP.0b013e31823d4f83}

\end{CSLReferences}

\endgroup


\end{document}
